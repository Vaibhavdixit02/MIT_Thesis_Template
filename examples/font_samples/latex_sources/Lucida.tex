% !TEX encoding = UTF-8 Unicode
% !BIB TS-program = biber
% !BIB program = biber 

% This file is MIT-Thesis.tex, a LaTeX template for formatting an MIT thesis with the mitthesis class.
%
% Version: 1.17, 2024/10/23
%
% Author: John H. Lienhard, copyright 2024. Reuse under the MIT license: https://ctan.org/license/mit 

% Documentation is here: https://ctan.org/pkg/mitthesis

%% Don't modify the \DocumentMetadata command unless you know what it does. 
%% If this command throws an "undefined" error, your latex system is out of date: try commenting this command out.
\DocumentMetadata{ 
	pdfstandard = a-2b,
	pdfversion  = 1.7,
	lang		= en-US,
%	 pdfversion  = 2.0,
%    pdfstandard = a-4,
}

%%%%%%%%%%%%%%%%%%%%%%%%%%%%%%%%%%%%%%%

\documentclass[fontset=lucida]{mitthesis}% fontset=newtx, fontset=libertine, fontset=newtx-sans-text, fontset=heros-stix2, fontset=stix2
%
% option [twoside]		gives facing-page behavior for printing; omitting twoside will eliminate even-numbered blank pages.
% option [lineno]	 	provides line numbers, as for editing
% option [mydesign] 	loads packages for color, title and list formats, margins, or captions: edit mydesign.tex to change defaults.
% option [fontset] is a keyvalue which can be:
%					 	for pdftex or unicode engines:  defaultfonts, libertine, lucida
%					 	for pdftex only: 				fira-newtxsf, newtx, newtx-sans-text
%						for unicode engines (luatex):	heros-stix2, stix2, termes, termes-stix2
%					 	if no key value is given, fonts default to CMR (pdftex) or LMR (unicode), i.e., "the LaTeX font".
%					 	You can edit the fontset files or you can write your own, myfonts.tex, and do [fontset=myfonts].
%						If you are using multiple languages, load the babel package in your fontset file, before the fonts.

%%%%%%%%% Packages used in sample chapters (not otherwise required) %%%%%%%

%% Package for code listing in Appendix A.
\usepackage{listings}%   documentation is here https://ctan.org/pkg/listings

%% Set chemical formulas nicely
\usepackage[version=4]{mhchem}%   documentation at https://ctan.org/pkg/mhchem

%% Latin filler used in Chapter 1, with a test for package version date (https://ctan.org/pkg/lipsum)
\usepackage{lipsum}
\IfPackageAtLeastTF{lipsum}{2021/09/20}{\setlipsum{auto-lang=false}}{}

%% Table related packages  

\usepackage{booktabs}% publication quality tables (https://ctan.org/pkg/booktabs)

\usepackage{array}% Additional options for column formats (https://ctan.org/pkg/array)

\usepackage{dcolumn}% For alignment of numbers on the decimal place (https://ctan.org/pkg/dcolumn) 
	\newcolumntype{d}[1]{D{.}{.}{#1}}% use with dcolumn package
	% syntax: d{x.y} where x is max number of digits before decimal and y is max number after.

% Package for multipage table in Appendix B.
\usepackage{longtable}% typeset multi-page tables (https://ctan.org/pkg/longtable)

%\usepackage{tabularx}% adjustable-width columns in tabular (https://ctan.org/pkg/tabularx)


%% Package for improved typography

\usepackage{microtype}% typographic fine-tuning, used in sample thesis committee page, but also acting globally on the text 


%%%%%%%%%  Graphics path (to figure files)  %%%%%%%%%%%%%%%%%%%%%%%%%%%%%%%%

%% Can set graphicspath to point to specific directories containing figures (the current directory is searched automatically)
%% For instance, to search a subdirectory of the current directory called "figures" and a parallel directory called "art", set:

% \graphicspath{ {figures/} {../art/} }% For details see: https://latexref.xyz/dev/latex2e.html#g_t_005cgraphicspath


%%%%%%%%%  Representative set-up for biblatex  %%%%%%%%%%%%%%%%%%%%%%%%%%%%%

%% Numerical citations of references
\usepackage[style=ext-numeric-comp,giveninits=true,maxbibnames=10,sorting=none]{biblatex}

%% IEEE style citations and references
% \usepackage[style=ieee,maxbibnames=10,sorting=none]{biblatex}% style=ext-numeric-comp,articlein=false,giveninits=true
%	 \DefineBibliographyStrings{english}{url= \textsc{url} ,  }% replaces the IEEE default "[Online]. Available" by "URL"

%% author-year style citations and references 
%% use \parencite, not \cite, when you want "(Author, year)"
%% The sample files are not set up to include parentheses.
% \usepackage[style=authoryear, maxbibnames=10]{biblatex} 


\addbibresource{mitthesis-sample.bib}%% <== change to YOUR bib file <= CHANGE

%% to avoid split urls and stretched white space, you can set the bibliography ragged-right:
%\appto{\bibsetup}{\raggedright}

%% biblatex is very powerful, and you can customize most aspects the reference list and citations to suit your needs.
%%   documentation is here: https://ctan.org/pkg/biblatex
%%   cheat sheet is here:   https://tug.ctan.org/info/biblatex-cheatsheet/biblatex-cheatsheet.pdf

%% To ensure citations are set, run Latex --> biblatex/biber --> Latex again

%%%%%%%%%%  Option to use natbib   %%%%%%%%%%%%%%%%%%%%%%%%%%%%%%%%%%%%%%%%%

%\RequirePackage[numbers,sort&compress]{natbib}
 
%%% add bibliography to table of contents
%\apptocmd{\bibliography}{\addcontentsline{toc}{chapter}{\protect\textbf{\bibname}}}{}{}

%%% You can use this to rename the bibliography section
%\renewcommand{\bibname}{References}

%%% To adjust space between bibliography items 
%\setlength\bibsep{4pt plus 1pt minus 1pt}
%   change 4pt to something else; don't drop last two lengths (they are stretchable "glue")


%%%%%%%%%%  Option for "double spacing" %%%%%%%%%%%%%%%%%%%%%%%%%%%%%%%%%%%%

%% Back in the typewriter era, double spaced lines were convenient for editing with a pencil. 
%% In typography, the separation between lines is called "leading", and it is usually set in 
%% proportion to the font size (i.e., when the font is loaded).  If you really feel the need 
%% to change the line separation, the most attractive results will be obtained by changing the
%% leading in proportion to the the current font size, rather than just doubling the space.

%% The setspace package provides a tool for changing line separation. Use these two commands here:
%
% \usepackage{setspace}%  documentation at https://ctan.org/pkg/setspace
% \setstretch{1.1}% you can choose some other value for the stretch of space between lines
%
%% Use one or more of the these commands *AFTER* the frontmatter
%
% \onehalfspacing
% \doublespacing
% \singlespacing  % will turn these effects off (you can use these anywhere in the document)

%% The best result is usually to stay with leading selected by the typographer who set up the font.


%%%%%%%%%%%  Metadata  %%%%%%%%%%%%%%%%%%%%%%%%%%%%%%%%%%%%%%%%%%%%%%%%%%%%%%%

% Most of the document metadata is created automatically. 
% The following items should be adjusted to match your work. <================= !!!!!!!!!!

\hypersetup{%
	pdfsubject={Template for writing MIT theses with the mitthesis class},
	% Change this to briefly state topic of your thesis 
% 
	pdfkeywords={Massachusetts Institute of Technology, MIT},
	% Add keywords that will help search engines and libraries to find your work.
	% Includes the name[s] of the author[s] 
	% (If you used \DocumentMetadata at line 14, you can just put "\CopyrightAuthor," for the names.)
%
	pdfurl={},
	% If you have a url for the thesis, put it here. Otherwise delete this.
	% (MIT Libraries will put your thesis in DSPACE with a persistent url after you submit it.)
%	
	pdfcontactemail={},
	% You can put a [permanent] email address into the metadata, if you like.
	% Otherwise delete this.
%
	pdfauthortitle={},
	% If you have a title, you can include it here.
}

%%%%%%%%%%%%  Math operators %%%%%%%%%%%%%%%%%%%%%%%%%%%%%%%%%%%%%%%%%%%%%%%%%%%%

% These commands declare two math operators, \erf{..} and \erfc{..} using mathtools
% note: * form produces automatic delimiter scaling; optional argument sets size manually, e.g. [\bigg]
% See Table 1.1 in Chapter 1

\DeclareMathOperator{\Erf}{erf}
\DeclareMathOperator{\Erfc}{erfc}
\DeclarePairedDelimiterXPP\erf[1]{\Erf\mkern1mu}(){}{#1}% increase to 2mu with stix2 font
\DeclarePairedDelimiterXPP\erfc[1]{\Erfc\mkern1mu}(){}{#1}

%%%%%%%%%%%%%%  End preamble %%%%%%%%%%%%%%%%%%%%%%%%%%%%%%%%%%%%%%%%%%%%%%%%%%%%%%%%%%%%%%%%%%%%%
%%%%%%%%%%%%%%%%%%%%%%%%%%%%%%%%%%%%%%%%%%%%%%%%%%%%%%%%%%%%%%%%%%%%%%%%%%%%%%%%%%%%%%%%%%%%%%%%%%

\begin{document}

%%% edit the following commands to match your thesis %%%%%%%%%%

\title{The Atomic Theory as Applied To Gases, with Some Experiments on the Viscosity of Air}

% \Author{Author full name}{Author department}[Author's first PREVIOUS degree][Author's second PREVIOUS degree][...
% Note that third, fourth, fifth, and sixth arguments are optional [] and may be omitted

% note on names: most of the following names are made up; Silas Holman was a physics professor at MIT in the 19th century.

\Author{Silas W. Holman}{Department of Physics}
% \Author{Luisa Hernández}{Department of Research}[B.S. Mechanical Engineering, UCLA, 2018][M.S. Stellar Interiors, Vulcan Science Academy, 2020]
% \Author{Thurston Howell III}{Department of Economics}[MBA, Ferengi School of Management, 2022]

% Use once for each degree fulfilled by thesis
% For two degrees from one department, leave the department argument blank for the second degree {}.
\Degree{Bachelor of Science in Physics}{Department of Physics}
%\Degree{Master of Science in Physics}{}
%\Degree{Bachelor of Science in Mechanical Engineering}{Department of Mechanical Engineering}

% If there is more than one supervisor, use the \Supervisor command for each.
\Supervisor{Edward C. Pickering}{Professor of Physics}
% \Supervisor{Edward C. Pickering}{Professor of Physics, and \\ \> Professor of Something Else}
% \Supervisor{Secunda Castor}{Professor of Research}
% \Supervisor{Quintus Castor}{Professor of Log Dams}

% Professor who formally accepts theses for your department (e.g., the Graduate Officer, Professor Sméagol,...)
% If more than one department, use more than once
\Acceptor{Tertius Castor}{Professor of Log Dams}{Graduate Officer, Department of Research} % \\ \> Third title}
% \Acceptor{Quarta Castor}{Professor of Lodge Building}{Graduate Officer, Department of Mechanical Engineering}
%%%  If you need to reduce vertical space, put the acceptor title in the second argument and leave the third blank, {}.
% \Acceptor{Primus Castor}{Professor and Undergraduate Officer, Department of Physics}{}

% Usage: \DegreeDate{Month}{year}
% Valid degree months are February, May, June, or September
\DegreeDate{June}{1876}

% Date that final thesis is submitted to department
\ThesisDate{May 18, 1876}


%%%%%%  Choose whether to have a CREATIVE COMMONS License  %%%%%%%%%%%%%%%%%%%%%%%%%%%%%%%%%%%%%%
%
% If you are using a cc license, put details of your cc license here. 
% Omit this command if you are not using a cc license.
%
\CClicense{CC BY-NC-ND 4.0}{https://creativecommons.org/licenses/by-nc-nd/4.0/}
%

%%%%%%%  Solutions for overflowing titlepage  %%%%%%%%%%%%%%%%%%%%%%%%%%%%%%%%%%%%%%%%%%%%%%%%%%%

% If your title page is overflowing (from too many names, degrees, etc.):
%
% (a) you can reduce the 12pt and 18pt skips between various blocks to 6pt with this command:
%
% \Tighten
%
% (b)  you can scale down the Signature block at the bottom with this command:
%
% \SignatureBlockSize{\small}  %or this one \SignatureBlockSize{\footnotesize}
%
% (c) you can put the acceptor name and title onto two lines, rather than three like this:
%
% \Acceptor{Tertius Castor}{Professor and Graduate Officer, Department of Research}{}
%
% (d) you can change the font size of the author name[s] with
%
%	\AuthorNameSize{\normalsize}
%
% (e) and you can omit any previous degrees from the title page, instead mentioning them in the biographical sketch

% Also, if you prefer to keep the text toward the top of the page with most white space at the bottom, you
% can use this command to squash all of the vertical glue (stretchy space) with this command:
%
% \Squash 
%
% This command is useful when the text has not already reach the bottom of the page, since the glue gets squashed automatically
% when the page is too full.

%%%%%%%%%%%%%%%%%%%%%%%%%%%%%%%%%%%%%%%%%%%%%%%%%%%%%%%%%%%%%%%%%%%%%%%%%%%%%%%%%%%%%%%%%%%%%%%%%

%%% Make titlepage
\maketitle

%%%%%%%%% Contents that you need to write follows! %%%%%%%%%%%%%%%%%%%%%%%%%%%%%%%%%%%%%%%%%%%%%%

% \includeonly{acknowledgments,biography,chapter1,chapter2,...,appendixa,...} 
%   for usage of includeonly, see https://latexref.xyz/_005cinclude-_0026-_005cincludeonly.html

%%% Frontmatter (write this material in the mentioned files)  %%%%%%%%%%%%%%%%%%%%%%%%%%%%%%%%%%%

% This page is optional. Edit the file committee_members.tex 
% Sample thesis committee page for mitthesis.cls
% Version 1.01, 2024/10/08
%
% This page is not required by the MIT Libraries, but some departments require it.
%
% Insert between title page and abstract page.
% Format this page in any way that you like.  
% Add supervisor titles, degrees, and departments as appropriate.

%%%%% FORMATTING COMMANDS %%%%%%%%%%%%%%%%%%

%% Format title
\NewDocumentCommand\CommitteePageTitle{m}{
	\vspace*{75pt}%36pt}
	\IfPackageLoadedTF{microtype}
		{\textls*{\Large\textbf{\MakeUppercase{#1}}}}
		{{\Large\textbf{\MakeUppercase{#1}}}}%
	\pdfbookmark[0]{#1}{Committee}%
	\vspace*{10pt}%
}
% \textls* produces additional letter separation (appropriate for capitalized display text),
% PROVIDED THAT \usepackage{microtype} has been loaded in the preamble. 
% The extra space added is 100/1000 em (adjustable, see package documentation).

%% Format committee member subheadings
\NewDocumentCommand\Role{m}{
	\vspace*{50pt}%25pt}
	\IfPackageLoadedTF{microtype}
		{\textls*{\large{\textsc{#1}}}}
		{{\large\textsc{#1}}}%
	\vspace*{12pt}%
}

%%%%%%%%%%%%%%%%%%%%%%%%%%%%%%%%%%%%%%%%%%%%%

\begin{flushright}

\CommitteePageTitle{Thesis committee}

\Role{Thesis Supervisor}

 \textbf{Marcus Gavius Apicius} \\
 {\itshape
 Professor of Cooking Arts \\
 Department of Food Science \\
 }

\Role{Thesis Readers}

 \textbf{Marie-Antoine Carême} \\
 {\itshape
   Professor of Haute Cuisine \\
   Department of Food Science \\[18pt]
 }

 \textbf{Julia Child}\\
 {\itshape
   Professor of French Cuisine \\
   Department of Food Science \\[18pt]
 }

 \textbf{Miles Gloriosus} \\
 {\itshape
   Professor of Personal Pronouns \\
   Department of Rhetoric \\
 }

\end{flushright}

\cleardoublepage


% The abstract environment creates all the required headings and footers. 
% You only need to the text of the abstract in the file abstract.tex
\begin{abstract}
	% From mitthesis package
% Version: 1.01, 2023/06/19
% Documentation: https://ctan.org/pkg/mitthesis
%
% The abstract environment creates all the required headers and footnote. 
% You only need to add the text of the abstract itself.
%
% Approximately 500 words or less; try not to use formulas or special characters
% If you don't want an initial indentation, do \noindent at the start of the abstract

This thesis develops novel computational and theoretical approaches for solving challenging continuous, nonlinear, and nonconvex optimization problems across statistics, machine learning, and optimal control. We make three primary contributions to address fundamental difficulties in non-convex optimization: First, we introduce Disciplined Geodesically Convex Programming (DGCP), which extends convexity verification to Riemannian manifolds and enables optimization on curved spaces with theoretical guarantees. We construct a comprehensive framework of rules and atoms for Cartan-Hadamard manifolds, with special attention to symmetric positive definite matrices. Second, we develop a GPU-accelerated hybrid optimization framework that combines the global exploration capability of Particle Swarm Optimization with the efficient local convergence of L-BFGS, demonstrating speedups on inverse problems with multiple local minima. Third, we propose an augmented Lagrangian method with stochastic inner optimizers such as Adam, bridging constrained optimization with modern machine learning techniques. Throughout, we emphasize practical implementation, providing optimized and composable libraries unified under the Optimization.jl ecosystem that makes these advanced methods accessible to practitioners across disciplines.% use \input rather than \include because we're inside an environment
\end{abstract}

%% acknowledgments.tex

% From mitthesis package
% Version: 1.02, 2024/06/19
% Documentation: https://ctan.org/pkg/mitthesis

\chapter*{Acknowledgments}
\pdfbookmark[0]{Acknowledgments}{acknowledgments}


Write your acknowledgments here.
% acknowledgments.tex (.tex extension is presumed by \include) 

%% biography.tex
%% This section is optional

% From mitthesis package
% Version: 1.02, 2024/06/19
% Documentation: https://ctan.org/pkg/mitthesis

\chapter*{Biographical Sketch}
\pdfbookmark[0]{Biographical Sketch}{biosketch}


Silas Whitcomb Holman was born in Harvard, Massachusetts on January 20, 1856. He received his S.B. degree in Physics from MIT in 1876, and then joined the MIT Department of Physics as an Assistant. He became Instructor in Physics in 1880, Assistant Professor in 1882, Associate Professor in 1885, and Full Professor in 1893. Throughout this period, he struggled with increasingly severe rheumatoid arthritis. At length, he was defeated, becoming Professor Emeritus in 1897 and dying on April 1, 1900.

Holman's light burned brilliantly before his tragic and untimely death. He published extensively in thermal physics, and authored textbooks on precision measurement, fundamental mechanics, and other subjects. He established the original Heat Measurements Laboratory. Holman was a much admired teacher among both his students and his colleagues. The reports of his department and of the Institute itself refer to him frequently in the 1880's and 1890's, in tones that gradually shift from the greatest respect to the deepest sympathy.

Holman was a student of Professor Edward C. Pickering, then head of the Physics department. Holman himself became second in command of Physics, under Professor Charles R. Cross, some years later. Among Holman's students, several went on to distinguish themselves, including: the astronomer George E. Hale ('90) who organized the Yerkes and Mt. Wilson observatories and who designed the 200 inch telescope on Mt. Palomar; Charles G. Abbot ('94), also an astrophysicist and later Secretary of the Smithsonian Institution; and George K. Burgess ('96), later Director of the Bureau of Standards. % biography.tex (optional, see https://libraries.mit.edu/distinctive-collections/thesis-specs/#format)

%%% Table of contents and lists of stuff (delete unused lists, i.e., if no tables or figures) %%%%%

\tableofcontents
\listoffigures
\listoftables

%%% Chapters of thesis  %%%%%%%%%%%%%%%%%%%%%%%%%%%%%%%%%%%%%%%%%%%%%%%%%%%%%%%%%%%%%%%%%%%%%%%%%%%

%% If you want to use "double spacing", you should start here...

 % From mitthesis package
% Version: 1.07, 2024/09/26
% Documentation: https://ctan.org/pkg/mitthesis


\chapter{Introduction}

\lipsum[1-2] Postremo aliquos futuros suspicor, qui me ad alias litteras vocent, genus hoc scribendi, etsi sit elegans, personae tamen et dignitatis esse negent~\cite{DKE1969,ww1920,kirk2288a,churchill1948,gibbs1863}.

\section[A section discussing the first issue: \(J/\psi\)]{A section discussing the first issue: \ifpdftex\(\bm{J}/\bm{\psi}\)\else{\(\symbfit{J/\psi}\)}\fi}


We begin with some ideas from the literature \cite{Fong2015,sharpe1}. 
\begin{equation}\label{eqn:1}
\frac{\partial}{\partial t}\left[\rho\bigl(e + \lvert\vec{u}\rvert^2\big/2\bigr)\right]  + \nabla\cdot\left[\rho\bigl(h + \lvert\vec{u}\rvert^2\big/2 \bigr)\vec{u}\right]
 ={}-\nabla \cdot \vec{q} +  \rho \vec{u}\cdot\vec{g}+ \frac{\partial}{\partial x_j}\bigl(d_{ji}u_i\bigr)
\end{equation}
 \lipsum[3]

\lipsum[4] And more citations~\cite{sharpe1,GSL}.  Then we write some more and include our citations~\cite{Swaminathan2017IDABRO,dlmf,amsmath}. The configuration is shown in Fig.~\ref{fig:golden2}.

%%%%%%%%%%%%%%%%%  begin figure  %%%%%%%%%%%%%%%%%%%%%%%%%%%
\begin{figure}[t]
% sample images are from mwe package, but should be found by latex in the tex tree w/o loading that package
\begin{subfigure}[c]{0.495\textwidth}
\centering{\includegraphics[alt={sample image},width=0.99\textwidth]{example-image-c}}%
\subcaption{\label{fig:golden}}
\end{subfigure}
%%%%%%%% don't leave a break here
\begin{subfigure}[c]{0.495\textwidth}
\centering{\includegraphics[alt={sample image},width=0.99\textwidth]{example-image-c}}%
\subcaption{\label{fig:golden2}}%
\end{subfigure}%
\caption{A figure with two subfigures: (a) first subfigure; (b) second subfigure.\label{fig:4}}
\end{figure}
%%%%%%%%%%%%%%%%%%%  end figure  %%%%%%%%%%%%%%%%%%%%%%%%%%%%%

\lipsum[4]

%% note use of \ref* here to avoid placing a nested link in the table of contents
\subsection[Subsection~eqn.~(\ref*{eqn:WT1})]{Subsection~eqn.~\eqref{eqn:WT1}}
\lipsum[5-6]

\subsubsection{A subsubsection}
\lipsum[7]

\begin{equation}\label{eqn:WT1}
L(\ifpdftex\bm{A}\else\symbfup{A}\fi) = \begin{pmatrix}
\dfrac\varphi{(\varphi_1,\varepsilon_1)}			& 0 												 & \hdotsfor{3} 							& 0 \\[4\jot]
\dfrac{\varphi k_{2,1}}{(\varphi_2,\varepsilon_1)}	& \dfrac\varphi{(\varphi_2,\varepsilon_2)}			 & 0 										& \hdotsfor{2} & 0 \\[4\jot]
\dfrac{\varphi k_{3,1}}{(\varphi_3,\varepsilon_1)}	& \dfrac{\varphi k_{3,2}}{(\varphi_3,\varepsilon_2)} & \dfrac\varphi{(\varphi_3,\varepsilon_3)}	& 0 & \hdotsfor{1} & 0 \\[\jot]
\vdots 												&  													 &  & \smash{\rotatebox{15}{$\ddots$}} &  & \vdots \\[\jot]
\dfrac{\varphi k_{n-1, 1}}{(\varphi_{n-1},\varepsilon_1)}		& \dfrac{\varphi k_{n-1, 2}}{(\varphi_{n-1},\varepsilon_2)} & \hdotsfor{1} & 
	\dfrac{\varphi k_{n-1,n-2}}{(\varphi_{n-1},\varepsilon_{n-2})}	& \dfrac{\varphi}{(\varphi_{n-1},\varepsilon_{n-1})} 		& 0 \\[4\jot]
\dfrac{\varphi k_{n,1}}{(\varphi_n,\varepsilon_1)}				& \dfrac{\varphi k_{n,2}}{(\varphi_n,\varepsilon_2)}		& \hdotsfor{2}	&
	\dfrac{\varphi k_{n,n-1}}{(\varphi_n,\varepsilon_{n-1})} 		& \dfrac{\varphi}{(\varphi_n,\varepsilon_n)}
\end{pmatrix}
\end{equation}

\section{Description our paradigm}\label{ch1:theidea}

\lipsum[8] No dissertation is complete without footnotes.\footnote{First footnote. $a_h = F_m$ See section~\ref{sec:stratified-flow}.}\footnote{Another interesting detail.}\footnote{And another really important idea to have in mind~\cite{reynolds1958,clauser56,lienhard2020,johnson1980,johnson1965,mpl}.} 

\begin{figure}[t]
% sample image is from mwe package, but should be found by latex in the tex tree w/o loading that package
\centering\includegraphics[alt={sample image},width=6.67cm]{example-image-b} 
\caption{Caption text\label{example-image-b}~\cite{GSL}.}
\end{figure}


\subsection{Conversion to a metaheuristic}

\lipsum[11-12] This concept is discussed further in section~\ref{sec:stratified-flow}, and Refs.~\cite{euler1740,fourier1822}.



\section{Other generalizations}

\subsection{The most general case}

\lipsum[7] And another citation, so that our sources will be unambiguous~\cite{montijano2014}.
\begin{gather}
\ce{x Na(NH4)HPO4 ->[\Delta] (NaPO3)_x + x NH3 ^ + x H2O} \\[0.5em]
\ce{^234_90Th -> ^0_-1$\beta${} + ^234_91Pa} \\[0.5em]
\ce{SO4^2- + Ba^2+ -> BaSO4 v} \\[0.5em]
\ce{Zn^2+
<=>[+ 2OH-][+ 2H+]
$\underset{\textrm{amphoteric hydroxide}}{\ce{Zn(OH)2 v}}$
<=>[+ 2OH-][+ 2H+]
$\underset{\textrm{tetrahydroxozincate}}{\ce{[Zn(OH)4]^2-}}$
}
\end{gather}
These examples of chemical formul\ae\ are copied directly from the documentation of the \texttt{mhchem} package, which was used to typeset them.

\section{Baroclinic generation of vorticity\label{sec:stratified-flow}}

Substitution of the particle acceleration and application Stokes theorem leads to the \textit{Kelvin-Bjerknes circulation theorem}, for
$\rho \neq \textrm{fn}(p)$:
\begin{align}
\frac{d\Gamma}{dt} &{}= \frac{d}{dt} \int_{\mathcal{C}} \mathbf{u} \cdot d\mathbf{r}\\
				   &{}= \int_{\mathcal{C}} \frac{D\mathbf{u}}{Dt} \cdot d\mathbf{r} + \underbrace{\int_{\mathcal{C}} \mathbf{u}\cdot d\biggl( \frac{d\mathbf{r}}{dt}\Biggr)}_{=\, 0} \\[-2pt]
                   &{}= \iint_{\mathcal{S}} \nabla \times \frac{D\mathbf{u}}{Dt}  \cdot d\mathbf{A}\\
                   &{}= \iint_{\mathcal{S}}  \nabla p \times \nabla \left( \frac{1}{\rho}\right) \cdot d\mathbf{A}
\end{align}

Baroclinic generation of vorticity accounts for the sea breeze and various other atmospheric currents in which temperature, rather than pressure, creates density gradients. Further, this phenomenon accounts for ocean currents in straits joining more and less saline seas, with surface currents flowing from the fresher to the saltier water and with bottom current going oppositely.

%% Nomenclature list is optional
%
%  This environment takes three optional arguments:
%		[1] adjust space between symbol and definition
%		[2] name (heading) of the nomenclature list
%		[3] level - can be "section" or "chapter" depending on whether you
%			have one nomenclature list for whole thesis or one for each
%			chapter. 
\begin{nomenclature}[2em][Nomenclature for Chapter~1][section]
\EntryHeading{Roman letters}
\entry{$\mathcal{C}$}{material curve}
\entry{$\mathbf{r}$}{material position [m]}
\entry{$\mathbf{u}$}{velocity [m s$^{-1}$]}
\EntryHeading{Greek letters}
\entry{$\Gamma$}{circulation [m$^2$ s$^{-1}$]}
\entry{$\rho$}{mass density [kg m$^{-3}$]}
\end{nomenclature}

%%%%%%%%%%%%%%% begin table %%%%%%%%%%%%%%%%%% 
\begin{table}[t]
\caption{The error function and complementary error function}\label{tab:1}%
\centering{%
\begin{tabular*}{0.8\textwidth}{@{\hspace*{1.5em}}@{\extracolsep{\fill}}ccc!{\hspace*{3.em}}ccc@{\hspace*{1.5em}}}
\\[-0.5em]
\toprule
\multicolumn{1}{@{\hspace*{1.5em}}c}{$x$\rule{0pt}{8pt}} &
\multicolumn{1}{c}{$\erf{x}$} &
\multicolumn{1}{c!{\hspace*{3.em}}}{$\erf{x}$} &
\multicolumn{1}{c}{$x$} &
\multicolumn{1}{c}{$\erfc{x}$} &
\multicolumn{1}{c@{\hspace*{1.5em}}}{$\erfc{x}$} \\ \midrule
0.00 & 0.00000 & 1.00000 & 1.10 & 0.88021 & 0.11980 \\
0.05 & 0.05637 & 0.94363 & 1.20 & 0.91031 & 0.08969 \\
0.10 & 0.11246 & 0.88754 & 1.30 & 0.93401 & 0.06599 \\
0.15 & 0.16800 & 0.83200 & 1.40 & 0.95229 & 0.04771 \\
0.20 & 0.22270 & 0.77730 & 1.50 & 0.96611 & 0.03389 \\
0.30 & 0.32863 & 0.67137 & 1.60 & 0.97635 & 0.02365 \\
0.40 & 0.42839 & 0.57161 & 1.70 & 0.98379 & 0.01621 \\
0.50 & 0.52050 & 0.47950 & 1.80 & 0.98909 & 0.01091 \\
0.60 & 0.60386 & 0.39614 & 1.82\makebox[0pt][l]{14} & 0.99000 & 0.01000 \\
0.70 & 0.67780 & 0.32220 & 1.90 & 0.99279 & 0.00721 \\
0.80 & 0.74210 & 0.25790 & 2.00 & 0.99532 & 0.00468 \\
0.90 & 0.79691 & 0.20309 & 2.50 & 0.99959 & 0.00041 \\
1.00 & 0.84270 & 0.15730 & 3.00 & 0.99998 & 0.00002 \\
\bottomrule
\end{tabular*}
}%
\end{table}
%%%%%%%%%%%%%%%% end table %%%%%%%%%%%%%%%%%%% 

% .tex extension is presumed
% \chapter{Background and Foundations}

This chapter provides the mathematical foundations for the optimization techniques developed in this thesis. We begin with a review of general optimization theory, emphasizing the challenges posed by non-convex problems. We then introduce the key concepts of Riemannian geometry and geodesic convexity that underpin our work on optimization over manifolds. Finally, we discuss the role of automatic differentiation in modern optimization algorithms and provide background on the specific techniques—particle swarm optimization, quasi-Newton methods, and augmented Lagrangian approaches—that form the basis of our hybrid optimization methods.

\section{Mathematical Optimization}

Mathematical optimization is concerned with finding the best element from a set of alternatives, as defined by an objective function. Formally, an optimization problem can be stated as:

\begin{equation}
\min_{x \in \mathcal{X}} f(x) \quad \text{subject to} \quad g_i(x) \leq 0, \, h_j(x) = 0
\end{equation}

where $f: \mathcal{X} \rightarrow \mathbb{R}$ is the objective function, $g_i: \mathcal{X} \rightarrow \mathbb{R}$ are inequality constraints, $h_j: \mathcal{X} \rightarrow \mathbb{R}$ are equality constraints, and $\mathcal{X}$ is the domain of the problem.

\subsection{Convex Optimization}

A particularly well-studied class of optimization problems are convex optimization problems, which satisfy the following conditions:

\begin{enumerate}
\item The objective function $f$ is convex:
\begin{equation}
f(\lambda x + (1-\lambda)y) \leq \lambda f(x) + (1-\lambda)f(y), \quad \forall x,y \in \mathcal{X}, \, \forall \lambda \in [0,1]
\end{equation}

\item The inequality constraint functions $g_i$ are convex.

\item The equality constraint functions $h_j$ are affine:
\begin{equation}
h_j(x) = a_j^T x + b_j
\end{equation}

\item The domain $\mathcal{X}$ is a convex set:
\begin{equation}
\lambda x + (1-\lambda)y \in \mathcal{X}, \quad \forall x,y \in \mathcal{X}, \, \forall \lambda \in [0,1]
\end{equation}
\end{enumerate}

Convex optimization problems possess several desirable properties:

\begin{itemize}
\item Local minima are also global minima, eliminating the need to distinguish between them.
\item Optimality conditions are necessary and sufficient, providing a clear certificate of optimality.
\item A rich theory of duality provides alternative approaches and insights into the problem structure.
\item Efficient algorithms with strong convergence guarantees exist for solving convex problems.
\end{itemize}

These properties have led to widespread adoption of convex optimization in various fields, from control systems to machine learning. However, many practical problems do not satisfy the convexity requirements, necessitating the study of non-convex optimization.

\subsection{Non-Convex Optimization}

Non-convex optimization problems arise when one or more of the convexity conditions are violated. These problems present significant challenges:

\begin{itemize}
\item Multiple local minima may exist, and distinguishing the global minimum from local minima is generally NP-hard.
\item Saddle points—where the gradient vanishes but the Hessian has both positive and negative eigenvalues—can trap optimization algorithms.
\item The objective function may have regions of high curvature or extended flat areas, leading to slow convergence or numerical instability.
\end{itemize}

Despite these challenges, non-convex optimization problems are ubiquitous in applications. The training of deep neural networks, parameter estimation in complex systems, and many problems in computational biology and materials science involve inherently non-convex objective functions.

\subsubsection{Local Optimization Methods}

Local optimization methods focus on finding local minima, typically by iteratively refining an initial guess. Gradient-based methods are the most common, using the gradient of the objective function to identify descent directions.

The simplest gradient-based method is gradient descent, which updates the current estimate $x_k$ according to:

\begin{equation}
x_{k+1} = x_k - \alpha_k \nabla f(x_k)
\end{equation}

where $\alpha_k > 0$ is the step size or learning rate. While simple to implement, gradient descent may converge slowly, particularly for ill-conditioned problems.

Newton's method incorporates second-order information by using the Hessian matrix:

\begin{equation}
x_{k+1} = x_k - [\nabla^2 f(x_k)]^{-1} \nabla f(x_k)
\end{equation}

This method can achieve quadratic convergence near a local minimum but requires computing and inverting the Hessian matrix, which may be prohibitively expensive for high-dimensional problems.

Quasi-Newton methods, such as BFGS (Broyden-Fletcher-Goldfarb-Shanno) and L-BFGS (Limited-memory BFGS), approximate the Hessian using information from previous iterations. These methods strike a balance between the computational efficiency of gradient descent and the convergence speed of Newton's method.

\subsubsection{Global Optimization Methods}

Global optimization methods attempt to find the global minimum by exploring the entire feasible region. These methods typically sacrifice the efficiency of local methods for the ability to escape local minima.

Simulated annealing, inspired by metallurgical annealing processes, allows uphill moves with a probability that decreases over time. This stochastic exploration helps the algorithm escape local minima and eventually settle in the global minimum.

Genetic algorithms maintain a population of candidate solutions and apply operations inspired by biological evolution—selection, crossover, and mutation—to generate new candidates. Over successive generations, the population evolves toward better solutions.

Particle swarm optimization, which we explore in depth in Chapter 4, uses a population of particles that move through the search space according to their own experience and the experience of the swarm as a whole.

\subsubsection{Stochastic Optimization Methods}

Stochastic optimization methods have gained prominence, particularly in machine learning, for their ability to handle large datasets and escape local minima. Stochastic gradient descent (SGD) approximates the gradient using a small batch of data points:

\begin{equation}
x_{k+1} = x_k - \alpha_k \nabla f_{\mathcal{B}_k}(x_k)
\end{equation}

where $\nabla f_{\mathcal{B}_k}$ is the gradient computed on batch $\mathcal{B}_k$. This approach reduces the computational cost per iteration and introduces noise that can help escape local minima.

Various extensions of SGD have been proposed, including momentum methods, which accelerate convergence by accumulating previous gradient information, and adaptive learning rate methods like Adam, which adjust the learning rate for each parameter based on historical gradient information.

\section{Riemannian Geometry and Manifold Optimization}

Many optimization problems naturally arise on spaces that are not flat Euclidean spaces but curved manifolds. Riemannian geometry provides the mathematical foundation for understanding these spaces and developing optimization algorithms that respect their intrinsic structure.

\subsection{Riemannian Manifolds}

A manifold $M$ is a topological space that locally resembles Euclidean space. More precisely, for each point $p \in M$, there exists an open neighborhood $U$ of $p$ and a homeomorphism $\phi: U \rightarrow V$ from $U$ to an open subset $V$ of $\mathbb{R}^n$. The pair $(U, \phi)$ is called a chart, and a collection of charts covering the entire manifold is called an atlas.

A Riemannian manifold $(M, g)$ is a differentiable manifold equipped with a Riemannian metric $g$, which defines an inner product on the tangent space at each point. The tangent space $T_p M$ at point $p \in M$ is the vector space consisting of all tangent vectors to $M$ at $p$. The Riemannian metric $g$ assigns to each point $p \in M$ an inner product $\langle \cdot, \cdot \rangle_p$ on $T_p M$.

The Riemannian metric allows us to define concepts like the length of curves on the manifold. If $\gamma: [a, b] \rightarrow M$ is a differentiable curve, its length is given by:

\begin{equation}
L(\gamma) = \int_a^b \sqrt{\langle \dot{\gamma}(t), \dot{\gamma}(t) \rangle_{\gamma(t)}} dt
\end{equation}

A geodesic is a curve that locally minimizes the distance between points. In the context of Riemannian geometry, geodesics generalize the concept of straight lines from Euclidean space.

\subsection{Cartan-Hadamard Manifolds}

A particularly important class of Riemannian manifolds for optimization are Cartan-Hadamard manifolds. These are complete, simply connected Riemannian manifolds with non-positive sectional curvature everywhere. Key examples include Euclidean spaces, hyperbolic spaces, and the space of symmetric positive definite matrices with the affine-invariant metric.

Cartan-Hadamard manifolds possess several properties that make them amenable to optimization:

\begin{itemize}
\item The exponential map $\exp_p: T_p M \rightarrow M$ at any point $p \in M$ is a diffeomorphism, which means that any two points on the manifold are connected by a unique geodesic.
\item Geodesics diverge at least as fast as in Euclidean space, which simplifies the analysis of geodesic convexity.
\item The logarithm map $\log_p: M \rightarrow T_p M$, which is the inverse of the exponential map, is well-defined on the entire manifold.
\end{itemize}

\subsection{Optimization on Manifolds}

Optimization on manifolds extends traditional optimization techniques to non-Euclidean spaces. The key challenge is to respect the intrinsic geometry of the manifold while performing the optimization.

A Riemannian optimization problem can be stated as:

\begin{equation}
\min_{x \in M} f(x)
\end{equation}

where $f: M \rightarrow \mathbb{R}$ is a smooth function and $M$ is a Riemannian manifold.

The gradient of $f$ at a point $p \in M$, denoted by $\text{grad} f(p)$, is the unique tangent vector in $T_p M$ that satisfies:

\begin{equation}
\langle \text{grad} f(p), v \rangle_p = D f(p)[v], \quad \forall v \in T_p M
\end{equation}

where $D f(p)[v]$ is the directional derivative of $f$ at $p$ in the direction $v$.

Riemannian optimization algorithms typically operate by iteratively moving along geodesics in descent directions. For example, Riemannian gradient descent updates the current estimate $x_k$ according to:

\begin{equation}
x_{k+1} = \exp_{x_k}(-\alpha_k \text{grad} f(x_k))
\end{equation}

where $\exp$ is the exponential map and $\alpha_k$ is the step size.

For computational efficiency, the exponential map is often approximated by a retraction, which is a first-order approximation that preserves key properties required for optimization.

\section{Geodesic Convexity}

Geodesic convexity extends the concept of convexity to Riemannian manifolds. It plays a crucial role in manifold optimization, analogous to the role of convexity in Euclidean optimization.

\subsection{Definition and Properties}

A set $S \subseteq M$ on a Riemannian manifold $M$ is geodesically convex if, for any two points $x, y \in S$, there exists a geodesic $\gamma: [0, 1] \rightarrow M$ such that $\gamma(0) = x$, $\gamma(1) = y$, and $\gamma(t) \in S$ for all $t \in [0, 1]$.

A function $f: S \rightarrow \mathbb{R}$ on a geodesically convex set $S \subseteq M$ is geodesically convex if for any geodesic $\gamma: [0, 1] \rightarrow S$, the composition $f \circ \gamma: [0, 1] \rightarrow \mathbb{R}$ is convex in the usual sense:

\begin{equation}
f(\gamma(t)) \leq (1-t)f(\gamma(0)) + tf(\gamma(1)), \quad \forall t \in [0, 1]
\end{equation}

Geodesically convex functions share many properties with convex functions on Euclidean spaces:

\begin{itemize}
\item Local minima are global minima.
\item The set of minimizers is geodesically convex (if non-empty).
\item For strictly geodesically convex functions, there is at most one minimizer.
\end{itemize}

However, geodesic convexity is generally more challenging to verify than Euclidean convexity, as it requires understanding the geodesic structure of the manifold. This challenge motivates our work on Disciplined Geodesically Convex Programming, which provides a systematic framework for verifying geodesic convexity.

\subsection{Geodesic Convexity on Specific Manifolds}

The manifold of symmetric positive definite matrices, denoted by $\mathcal{P}_d$, is of particular interest in many applications. With the affine-invariant metric, $\mathcal{P}_d$ becomes a Cartan-Hadamard manifold.

For any two matrices $A, B \in \mathcal{P}_d$, the unique geodesic connecting them is given by:

\begin{equation}
\gamma(t) = A^{1/2} \left( A^{-1/2} B A^{-1/2} \right)^t A^{1/2}, \quad t \in [0, 1]
\end{equation}

A function $f: \mathcal{P}_d \rightarrow \mathbb{R}$ is geodesically convex if for any $A, B \in \mathcal{P}_d$ and $t \in [0, 1]$:

\begin{equation}
f\left( A^{1/2} \left( A^{-1/2} B A^{-1/2} \right)^t A^{1/2} \right) \leq (1-t)f(A) + tf(B)
\end{equation}

Several important functions on $\mathcal{P}_d$ are known to be geodesically convex, including:

\begin{itemize}
\item The log-determinant function: $f(X) = -\log \det(X)$
\item The trace function: $f(X) = \text{tr}(X)$
\item The Riemannian distance function: $f(X) = d^2(X, Y)$ for fixed $Y \in \mathcal{P}_d$
\end{itemize}

\section{Automatic Differentiation}

Optimization algorithms frequently require derivatives of the objective function and constraints. Traditionally, these derivatives were either computed analytically or approximated using finite differences. Automatic differentiation (AD) provides an alternative that combines the accuracy of analytical derivatives with the convenience of automatic computation.

\subsection{Forward and Reverse Mode}

Automatic differentiation is based on the chain rule of calculus and operates by decomposing a function into a sequence of elementary operations. There are two primary modes of automatic differentiation: forward mode and reverse mode.

Forward mode AD computes the derivative of intermediate variables with respect to the input variables as the computation progresses forward. For a function $f: \mathbb{R}^n \rightarrow \mathbb{R}^m$, forward mode AD computes the Jacobian-vector product $J \cdot v$ for a specified vector $v$, where $J$ is the Jacobian matrix of $f$.

Reverse mode AD first computes the output of the function, then propagates derivative information backward through the computation graph. For a function $f: \mathbb{R}^n \rightarrow \mathbb{R}^m$, reverse mode AD computes the vector-Jacobian product $v^T \cdot J$ for a specified vector $v$.

The choice between forward and reverse mode typically depends on the dimensions of the input and output spaces. Forward mode is more efficient when $n < m$ (few inputs, many outputs), while reverse mode is more efficient when $n > m$ (many inputs, few outputs). In optimization, where the objective function typically maps from $\mathbb{R}^n$ to $\mathbb{R}$, reverse mode is often the more efficient choice.

\subsection{AD Systems in Julia}

The Julia programming language offers several powerful AD systems, which form the basis of the differentiation capabilities in our optimization framework:

\begin{itemize}
\item \textbf{ForwardDiff.jl} implements forward-mode AD using dual numbers.
\item \textbf{ReverseDiff.jl} implements reverse-mode AD with a tape-based approach.
\item \textbf{Zygote.jl} provides source-to-source reverse-mode AD, which can generate efficient derivative code.
\item \textbf{Enzyme.jl} performs AD at the LLVM IR level, enabling high-performance differentiation of compiled code.
\end{itemize}

These systems offer different trade-offs in terms of performance, memory usage, and compatibility with Julia's features. Our optimization framework, described in Chapter 6, allows users to select the most appropriate AD system for their specific problem.

\section{Specific Optimization Techniques}

This section introduces the specific optimization techniques that form the basis of the hybrid methods developed in this thesis.

\subsection{Particle Swarm Optimization}

Particle Swarm Optimization (PSO) is a population-based stochastic optimization technique inspired by the social behavior of bird flocking or fish schooling. PSO maintains a population of candidate solutions, called particles, that move through the search space according to simple mathematical formulas.

Each particle $i$ has a position $x_i$ and a velocity $v_i$. The position represents a candidate solution, while the velocity determines how the position changes over time. The particles are guided by their own best known position, $p_i$, and the best position found by any particle in the swarm, $g$.

The update equations for particle $i$ at iteration $k$ are:

\begin{align}
v_i^{k+1} &= w v_i^k + c_1 r_1 (p_i - x_i^k) + c_2 r_2 (g - x_i^k) \\
x_i^{k+1} &= x_i^k + v_i^{k+1}
\end{align}

where $w$ is the inertia weight, $c_1$ and $c_2$ are acceleration coefficients, and $r_1$ and $r_2$ are random numbers in $[0, 1]$.

PSO has several advantages for global optimization:

\begin{itemize}
\item It does not require gradient information, making it suitable for non-differentiable or noisy objective functions.
\item It can explore multiple regions of the search space simultaneously, increasing the likelihood of finding the global optimum.
\item It has relatively few parameters to tune and is straightforward to implement.
\end{itemize}

However, PSO also has limitations:

\begin{itemize}
\item It may converge prematurely to local optima, especially in high-dimensional spaces.
\item The quality of the solution depends on the initial distribution of particles.
\item It typically requires more function evaluations than gradient-based methods.
\end{itemize}

In Chapter 4, we address these limitations by developing a hybrid approach that combines PSO with quasi-Newton methods.

\subsection{Quasi-Newton Methods}

Quasi-Newton methods approximate the Hessian matrix or its inverse using information from successive iterations, avoiding the computational cost of explicitly computing the Hessian. These methods are particularly effective for smooth, unconstrained optimization problems.

\subsubsection{BFGS Method}

The BFGS method constructs an approximation to the inverse Hessian, $H_k \approx [\nabla^2 f(x_k)]^{-1}$, using the secant condition:

\begin{equation}
H_{k+1} s_k = y_k
\end{equation}

where $s_k = x_{k+1} - x_k$ is the step and $y_k = \nabla f(x_{k+1}) - \nabla f(x_k)$ is the change in gradient.

The BFGS update formula is:

\begin{equation}
H_{k+1} = \left(I - \frac{s_k y_k^T}{y_k^T s_k}\right) H_k \left(I - \frac{y_k s_k^T}{y_k^T s_k}\right) + \frac{s_k s_k^T}{y_k^T s_k}
\end{equation}

This update ensures that $H_{k+1}$ remains positive definite if $H_k$ is positive definite and $y_k^T s_k > 0$, which is guaranteed when the objective function is convex.

\subsubsection{L-BFGS Method}

The Limited-memory BFGS (L-BFGS) method is a variant of BFGS designed for high-dimensional problems. Instead of storing the full approximate inverse Hessian, L-BFGS stores a limited number of vector pairs $\{s_i, y_i\}$ and uses them to implicitly represent the approximation.

L-BFGS performs the same update as BFGS but discards older vector pairs once a specified maximum number is reached. This approach significantly reduces memory requirements while still capturing the dominant curvature information.

The L-BFGS method is particularly effective for machine learning applications, where the objective function often involves a large number of parameters.

\subsection{Augmented Lagrangian Methods}

Augmented Lagrangian methods address constrained optimization problems by combining the Lagrangian with a quadratic penalty term. For a problem with equality constraints:

\begin{equation}
\min_{x \in \mathbb{R}^n} f(x) \quad \text{subject to} \quad h(x) = 0
\end{equation}

the augmented Lagrangian is:

\begin{equation}
L_A(x, \lambda, \mu) = f(x) + \lambda^T h(x) + \frac{\mu}{2} \|h(x)\|^2
\end{equation}

where $\lambda$ is the Lagrange multiplier and $\mu > 0$ is the penalty parameter.

The method proceeds by alternately minimizing $L_A$ with respect to $x$ and updating the Lagrange multipliers:

\begin{align}
x^{k+1} &= \arg\min_x L_A(x, \lambda^k, \mu^k) \\
\lambda^{k+1} &= \lambda^k + \mu^k h(x^{k+1})
\end{align}

If necessary, the penalty parameter $\mu^k$ is increased to ensure constraint satisfaction.

Augmented Lagrangian methods can be extended to handle inequality constraints by introducing slack variables or using specialized update rules. They have several advantages for constrained optimization:

\begin{itemize}
\item They typically exhibit faster convergence than penalty methods.
\item They avoid the ill-conditioning associated with large penalty parameters.
\item They can handle constraints directly, without requiring feasible iterates.
\end{itemize}

In Chapter 5, we extend this approach by incorporating stochastic optimizers for the inner minimization problem, which is particularly beneficial for large-scale machine learning applications.

\section{Summary}

This chapter has provided the mathematical foundations for the optimization techniques developed in this thesis. We have introduced the key concepts of convex and non-convex optimization, Riemannian geometry and geodesic convexity, automatic differentiation, and specific optimization techniques including particle swarm optimization, quasi-Newton methods, and augmented Lagrangian approaches.

These concepts form the basis for our contributions in the subsequent chapters: Disciplined Geodesically Convex Programming, GPU-accelerated hybrid optimization, and augmented Lagrangian methods with stochastic inner optimizers. By building on these foundations, we develop novel approaches to non-convex optimization that address the challenges of multiple local minima, manifold constraints, and computational efficiency.

%\chapter{Disciplined Geodesically Convex Programming}

This chapter introduces Disciplined Geodesically Convex Programming (DGCP), a framework that extends convexity verification to Riemannian manifolds. We begin by motivating the need for such a framework, then present the theoretical foundations of DGCP, including rules for preserving geodesic convexity under various operations. We then focus on the specific case of the symmetric positive definite manifold, introducing a comprehensive set of atoms and rules for this important domain. Finally, we present our implementation of DGCP in Julia and demonstrate its application to several important problems in statistics, machine learning, and optimization.

\section{Motivation and Background}

Nonlinear programming plays a fundamental role in data science, machine learning, and engineering. These problems involve optimization tasks with nonlinear objectives and/or nonlinear constraints, which can be challenging to solve efficiently. While convex programming—the subset of nonlinear programming where both the objective and constraints are convex—has been extensively studied and offers strong guarantees, many important applications fall outside this restrictive setting.

Consider the computation of statistical estimators such as Tyler's M-estimator and related estimators \cite{Tyler1987, Wiesel2012, Ollila2014}, optimistic likelihood estimation \cite{Nguyen2019}, and certain Wasserstein bounds on entropy \cite{Courtade2017}. Similarly, matrix-valued operations that arise in machine learning approaches including robust subspace recovery \cite{Zhang2016}, matrix barycenter problems \cite{Bhatia1997}, and learning Determinantal Point Processes \cite{Mariet2015} all involve non-convex optimization problems in the Euclidean sense.

However, these seemingly non-convex problems often possess "hidden" convexity when viewed through a geometric lens. While their objectives and constraints may be non-convex in the Euclidean sense, they are convex with respect to a different Riemannian metric. This observation motivates the need for tools that can automatically detect and leverage this geodesic convexity.

\subsection{Disciplined Convex Programming}

In the Euclidean setting, Disciplined Convex Programming (DCP) was introduced by Grant et al. \cite{Grant2006} as a framework for automating the verification of convexity in optimization problems. DCP works by decomposing the objective function or constraints into basic convex functions (atoms) using convexity-preserving compositions and transformations (rules). The framework has been implemented in libraries such as CVX \cite{Diamond2016}, making it accessible to practitioners.

More recently, the DCP framework has been extended to handle log-log convex programs \cite{Agrawal2019} and quasi-convex programs \cite{Agrawal2020}. However, no extensions to the geodesically convex setting existed prior to our work.

\subsection{Geodesic Convexity and Its Importance}

Geodesic convexity extends the notion of convexity to functions defined on manifolds. A function is geodesically convex if its restriction to any geodesic is convex. This concept is particularly important for optimization on manifolds, as geodesically convex functions possess many of the desirable properties of convex functions, such as the uniqueness of local minima.

For many applications, formulating problems in terms of geodesic convexity can transform seemingly non-convex problems into tractable ones with theoretical guarantees. However, verifying geodesic convexity manually requires expertise in differential geometry and can be error-prone. A framework that automates this verification process would significantly broaden the applicability of geodesic convexity in practical optimization.

\section{Theoretical Foundations of DGCP}

In this section, we present the theoretical foundations of Disciplined Geodesically Convex Programming. We focus on Cartan-Hadamard manifolds, which are complete, simply connected Riemannian manifolds with non-positive sectional curvature. These manifolds provide a natural setting for geodesic convexity, as they share many properties with Euclidean space while allowing for more general geometries.

\subsection{Rules for Cartan-Hadamard Manifolds}

We begin by establishing general rules for preserving geodesic convexity on Cartan-Hadamard manifolds. These rules form the foundation of our DGCP framework, allowing us to verify geodesic convexity of complex functions by decomposing them into simpler components.

\begin{proposition}[Maximum of Geodesically Convex Functions]
Let $(M, d)$ be a Cartan-Hadamard manifold. Suppose $S \subseteq M$ is a g-convex subset. Furthermore, suppose $f_i: S \rightarrow \mathbb{R}$ are g-convex for $i = 1, \ldots, n$. Then the function
\begin{equation}
f(X) = \max_{i \in \{1, \ldots, n\}} f_i(X)
\end{equation}
is also g-convex.
\end{proposition}

\begin{proposition}[Conic Combination of Geodesically Convex Functions]
Let $(M, d)$ be a Cartan-Hadamard manifold. Suppose $S \subseteq M$ is a g-convex subset. Furthermore, suppose $f_i: S \rightarrow \mathbb{R}$ are g-convex for $i = 1, \ldots, n$. Then the function
\begin{equation}
f(X) = \sum_{i=1}^n \alpha_i f_i(X)
\end{equation}
for $\alpha_1, \ldots, \alpha_n \geq 0$ is also g-convex.
\end{proposition}

\begin{remark}
In the setting of Cartan-Hadamard manifolds, the property of taking the maximum can be generalized to an arbitrary collection of g-convex sets. That is, for an arbitrary collection of g-convex functions $\{f_i\}_{i \in I}$, indexed by $I$, the map $X \mapsto \sup_{i \in I} f_i(X)$ is g-convex. This follows from the fact that a function $f$ is g-convex if and only if its epigraph is g-convex and the fact that the epigraph of the supremum of a collection of functions is the intersection of the epigraphs of each function in such a collection. Finally, the intersection of g-convex sets is g-convex for Cartan-Hadamard manifolds.
\end{remark}

Another important rule concerns the composition of geodesically convex functions with convex functions:

\begin{proposition}[Composition with Convex Functions]
Let $(M, d)$ be a Cartan-Hadamard manifold and $S \subset M$ g-convex. Suppose $f: S \rightarrow \mathbb{R}$ is g-convex. If $h: \mathbb{R} \rightarrow \mathbb{R}$ is non-decreasing and Euclidean convex, then $h \circ f: S \rightarrow \mathbb{R}$ is g-convex.
\end{proposition}

We can extend this result to various combinations of convexity and monotonicity:

\begin{corollary}[Scalar Composition Rules]
Let $(M, d)$ be a Cartan-Hadamard manifold and $S \subset M$ g-convex.
\begin{enumerate}
\item Let $f: S \rightarrow \mathbb{R}$ be geodesically concave. If $h: \mathbb{R} \rightarrow \mathbb{R}$ is non-increasing and convex, then $h \circ f$ is geodesically convex on $S$.
\item Let $f: S \rightarrow \mathbb{R}$ be geodesically concave. If $h: \mathbb{R} \rightarrow \mathbb{R}$ is non-decreasing and concave, then $h \circ f$ is geodesically concave on $S$.
\item Let $f: S \rightarrow \mathbb{R}$ be geodesically convex. If $h: \mathbb{R} \rightarrow \mathbb{R}$ is non-increasing and concave, then $h \circ f$ is geodesically convex on $S$.
\end{enumerate}
\end{corollary}

\begin{example}
If $f: S \rightarrow \mathbb{R}$ is g-convex with respect to the canonical Riemannian metric, then $\exp(f(x))$ is g-convex and $-\log(-f(x))$ is g-convex on $\{x: f(x) < 0\}$. If $f$ is non-negative and $p \geq 1$, then $f(x)^p$ is g-convex.
\end{example}

\subsection{Atoms for Cartan-Hadamard Manifolds}

In addition to rules that preserve geodesic convexity, we need a set of basic geodesically convex functions, or "atoms," from which we can build more complex functions. Here we present some fundamental atoms for general Cartan-Hadamard manifolds.

\begin{example}[Intrinsic Distance]
Let $(M, d)$ be a Cartan-Hadamard manifold. The following functions $f: M \rightarrow \mathbb{R}$ are geodesically convex.
\begin{enumerate}
\item Let $y \in M$. Then the intrinsic distance to $y$ given by $f(x) = d(x, y)$ is geodesically convex. More generally, $f(x) = d^p(x, y)$ is geodesically convex for $p \geq 1$. Furthermore, let $\{x_i\}_{i=1}^n \subseteq M$ and $w_1, \ldots, w_n > 0$ such that $\sum_{i=1}^n w_i = 1$. Then
\begin{equation}
f(x) = \sum_{i=1}^n w_i d^p(x, x_i)
\end{equation}
is geodesically convex for $p \geq 1$.
\item Let $F: M \rightarrow M$ be an isometry. Then the function
\begin{equation}
f_F(x) = d(x, Fx)
\end{equation}
is geodesically convex.
\end{enumerate}
\end{example}

These atoms, combined with the rules presented earlier, allow us to verify the geodesic convexity of a wide range of functions on general Cartan-Hadamard manifolds. In the next section, we focus on the specific case of the symmetric positive definite manifold, which is of particular interest in many applications.

\section{DGCP for Symmetric Positive Definite Matrices}

The manifold of symmetric positive definite matrices, denoted as $\mathcal{P}_d$, is a Cartan-Hadamard manifold when equipped with the affine-invariant metric. This manifold appears in numerous applications, from covariance matrix estimation to diffusion tensor imaging, making it a natural focus for our DGCP framework.

\subsection{Geometry of the Symmetric Positive Definite Manifold}

The manifold $\mathcal{P}_d$ consists of all $d \times d$ real symmetric matrices with strictly positive eigenvalues:
\begin{equation}
\mathcal{P}_d := \{X \in \mathbb{R}^{d \times d} : X^T = X, X \succ 0\}
\end{equation}

When equipped with the canonical affine-invariant metric, which defines the inner product on the tangent space as
\begin{equation}
\langle A, B \rangle_X = \text{tr}(X^{-1} A X^{-1} B), \quad X \in \mathcal{P}_d, A, B \in T_X(\mathcal{P}_d) = \mathcal{S}_d,
\end{equation}
where $\mathcal{S}_d$ is the space of $d \times d$ real symmetric matrices, $\mathcal{P}_d$ becomes a Cartan-Hadamard manifold.

The geodesic connecting two matrices $A, B \in \mathcal{P}_d$ has the explicit form
\begin{equation}
\gamma(t) = A^{1/2} (A^{-1/2} B A^{-1/2})^t A^{1/2}, \quad 0 \leq t \leq 1.
\end{equation}

The Riemannian distance on $\mathcal{P}_d$ is given by
\begin{equation}
\delta_R(A, B) = \|{\log(A^{-1/2} B A^{-1/2})}\|_F,
\end{equation}
which corresponds to the length of the geodesic connecting $A$ and $B$.

\subsection{Rules for Symmetric Positive Definite Matrices}

In addition to the general rules for Cartan-Hadamard manifolds, the specific geometry of $\mathcal{P}_d$ allows us to establish additional rules for preserving geodesic convexity. These rules are particularly useful for verifying geodesic convexity of functions that arise in applications involving symmetric positive definite matrices.

We begin by introducing the Löwner order, which provides a partial ordering on symmetric matrices:

\begin{definition}[Löwner Order]
For $A, B \in \mathcal{P}_d$, we write $A \succ B$ when $A - B \in \mathcal{P}_d$. Similarly, we write $A \succeq B$ whenever $A - B$ is symmetric positive semi-definite.
\end{definition}

We say a function $f: \mathcal{P}_d \rightarrow \mathbb{R}$ is increasing if $f(A) \geq f(B)$ whenever $A \succeq B$.

\begin{definition}[Positive Linear Map]
A linear map $\Phi: \mathcal{P}_d \rightarrow \mathcal{P}_m$ is positive when $\Phi(A) \succeq 0$ for all $A \in \mathcal{P}_d$. We say that $\Phi$ is strictly positive when $A \succ 0$ implies that $\Phi(A) \succ 0$.
\end{definition}

A key result for our framework is that strictly positive linear operators preserve geodesic convexity:

\begin{proposition}[Geodesic Convexity of Strictly Positive Linear Operators]
Let $\Phi(X)$ be a strictly positive linear operator from $\mathcal{P}_d$ to $\mathcal{P}_m$. Then $\Phi(X)$ is g-convex with respect to the Löwner order on $\mathcal{P}_m$ over $\mathcal{P}_d$ with respect to the canonical Riemannian inner product $g_X(U, V) := \text{tr}(X^{-1} U X^{-1} V)$. In other words, for any geodesic $\gamma: [0, 1] \rightarrow \mathcal{P}_d$, we have
\begin{equation}
\Phi(\gamma(t)) \preceq (1 - t)\Phi(\gamma(0)) + t\Phi(\gamma(1)) \quad \forall t \in [0, 1].
\end{equation}
\end{proposition}

This result leads to several important examples of geodesically convex maps:

\begin{example}[Strictly Positive Linear Operators]
Let $Y \in \mathcal{P}_d$ be fixed. The following maps are g-convex with respect to the canonical Riemannian metric on $\mathcal{P}_d$:
\begin{enumerate}
\item $X \mapsto \text{tr}(X)$
\item $X \mapsto Y^T X Y$ for $Y \in \mathbb{R}^{d \times k}$
\item $X \mapsto \text{Diag}(X) := \sum_j X_{jj} E_{jj}$, where $E_{jj}$ is the $d \times d$ matrix with 1 in the $(j, j)$-th element and 0 elsewhere
\item Let $M \succeq 0$ and $M$ has no zero rows. The function $\Phi(X) = M \odot X$ where $\odot$ denotes the Hadamard product is a strictly positive linear operator and hence g-convex.
\end{enumerate}
\end{example}

Another important result relates to the composition of positive linear maps with the log-determinant function:

\begin{proposition}[Log-Determinant Composition]
Let $\Phi(X): \mathcal{P}_d \rightarrow \mathcal{P}_m$ be a strictly positive linear operator. Then $\log \det(\Phi(X))$ is g-convex on $\mathcal{P}_d$ with respect to the metric $g_X(U, V) := \text{tr}(X^{-1} U X^{-1} V)$.
\end{proposition}

An interesting property of geodesic convexity on $\mathcal{P}_d$ is that it is preserved under matrix inversion:

\begin{proposition}[Geodesic Convexity and Matrix Inversion]
Let $f: \mathcal{P}_d \rightarrow \mathbb{R}$ be g-convex. Then $g(X) = f(X^{-1})$ is also g-convex.
\end{proposition}

Applying these results yields a rich variety of geodesically convex functions:

\begin{example}
The following maps are g-convex with respect to the canonical Riemannian metric:
\begin{enumerate}
\item $X \mapsto \log \det\left(\frac{X+Y}{2}\right)$ for fixed $Y \in \mathcal{P}_d$
\item $X \mapsto \log \det(X^r Y)$ for fixed $Y \in \mathcal{P}_d$ and $r \in \{-1, 1\}$
\item $X \mapsto \log \det\left(\sum_{i=1}^n Y_i X^r Y_i^T\right)$ for $\{Y_1, \ldots, Y_n\} \subseteq \mathcal{P}_d$ and $r \in \{-1, 1\}$
\end{enumerate}
\end{example}

A particularly important special case is the log-quadratic form:

\begin{example}[Log-Quadratic Form]
Let $y_i \in \mathbb{R}^d \setminus \{0\}$ for $i = 1, \ldots, m$. The function
\begin{equation}
X \mapsto \log\left(\sum_{i=1}^m y_i^T X y_i\right)
\end{equation}
is g-convex with respect to the canonical Riemannian metric.
\end{example}

We also have results for functions of eigenvalues:

\begin{example}
The following maps are g-convex:
\begin{enumerate}
\item $g(X) = \sum_{i=1}^k \lambda_i^{\downarrow}(X^{-1})$ for $k = 1, \ldots, d$
\item $g(X) = \sum_{i=1}^k \log\left(\lambda_i^{\downarrow}(X^{-1})\right)$ for $k = 1, \ldots, d$
\item $g(X) = \log \det\left(\frac{X^{-1}+Y}{2}\right)$ for fixed $Y \in \mathcal{P}_d$
\end{enumerate}
\end{example}

We can generalize the result on log-determinant composition to a broader class of functions:

\begin{proposition}[General Composition of Positive Maps]
Let $h: \mathcal{P}_d \rightarrow \mathbb{R}$ be non-decreasing and g-convex. Let $r \in \{-1, 1\}$ and let $\Phi$ be a positive linear map. Then $\varphi(X) = h(\Phi(X^r))$ is g-convex with respect to the canonical Riemannian metric.
\end{proposition}

We can extend this further to positive affine operators:

\begin{definition}[Positive Affine Operator]
Let $B \succeq 0$ be a fixed symmetric positive semidefinite matrix and $\Phi: \mathcal{P}_d \rightarrow \mathcal{P}_d$ be a positive linear operator. Then the function $\varphi: \mathcal{P}_d \rightarrow \mathcal{P}_d$ defined by
\begin{equation}
\varphi(X) = \Phi(X) + B
\end{equation}
is a positive affine operator.
\end{definition}

\begin{proposition}[Geodesic Convexity of Positive Affine Maps]
Let $\varphi(X) = \Phi(X) + B$ where $\Phi(X)$ is a positive linear map and $B \succeq 0$. Let $f: \mathcal{P}_d \rightarrow \mathcal{P}_m$ be g-convex and monotonically increasing, i.e., $f(X) \preceq f(Y)$ whenever $X \preceq Y$. Then the function $g(X) = f(\varphi(X))$ is g-convex.
\end{proposition}

These rules yield additional examples of geodesically convex functions:

\begin{example}
Let $B \succeq 0$ and $Y_i \in \mathcal{P}_d$ for $i = 1, \ldots, n$ be fixed matrices.
\begin{enumerate}
\item $X \mapsto \text{tr}\left(B + \sum_i Y_i^T X^r Y_i\right)^\alpha$ is g-convex
\item $X \mapsto \log \det\left(B + \sum_i Y_i^T X Y_i\right)$ is g-convex
\item Let $M \succeq 0$. The map $X \mapsto \log \det(B + X \odot M)$ is g-convex
\end{enumerate}
\end{example}

Finally, we have a powerful result for functions of eigenvalues:

\begin{theorem}[Geodesic Convexity of Spectral Functions]
If $f: \mathbb{R} \rightarrow \mathbb{R}$ is Euclidean convex, then the function $\varphi(X) = \sum_{i=1}^k f\left(\log \lambda_i^{\downarrow}(X)\right)$ is g-convex for each $1 \leq k \leq d$ where $\lambda_i^{\downarrow}(X)$ denotes the ordered spectrum of $X$, i.e., $\lambda_1^{\downarrow}(X) \geq \lambda_2^{\downarrow}(X) \cdots \geq \lambda_d^{\downarrow}(X)$. Moreover, if $h: \mathbb{R} \rightarrow \mathbb{R}$ is non-decreasing and Euclidean convex, then $\varphi(X) = \sum_{i=1}^k h(|\log \lambda_i^{\downarrow}(X)|)$ is g-convex for each $1 \leq k \leq n$.
\end{theorem}

\subsection{Atoms for Symmetric Positive Definite Matrices}

In addition to the rules for preserving geodesic convexity, our DGCP framework includes a foundational set of geodesically convex functions, or "atoms," defined on the manifold of symmetric positive definite matrices. These atoms serve as building blocks for constructing more complex geodesically convex functions through the application of the rules presented earlier.

In our framework, each atom has a designated curvature (g-convex, g-concave, or g-linear) and monotonicity property (GIncreasing, GDecreasing, or GAnyMono). The monotonicity property is defined with respect to the Löwner order:

\begin{definition}
A function $f: \mathcal{P}_d \rightarrow \mathcal{P}_d$ is GIncreasing if it satisfies $f(A) \succeq f(B)$ whenever $A \succeq B$.
\end{definition}

We now present our basic set of DGCP atoms, categorized as scalar-valued and matrix-valued.

\subsubsection{Scalar-valued Atoms}

\paragraph{Log Determinant.} $\text{logdet}(X)$ represents the log-determinant function $\log \det: \mathcal{P}_d \rightarrow \mathbb{R}_{++}$. This is an atom that is GLinear (i.e., both g-convex and g-concave) and GIncreasing. It is concave in the Euclidean setting.

\paragraph{Trace.} $\text{tr}(X)$ sums the diagonal entries of a matrix. It has GConvex curvature and is GIncreasing. It is affine in the Euclidean setting.

\paragraph{Sum of Entries.} $\text{sum}(X)$ returns $\sum_{i,j=1}^d X_{ij}$. It has GConvex curvature and is GIncreasing. It is affine in the Euclidean setting.

\paragraph{S-Divergence.} $\text{sdivergence}(X,Y)$ is defined as
\begin{equation}
\text{sdivergence}(X,Y) := \log \det\left(\frac{X + Y}{2}\right) - \frac{1}{2}\log \det(XY).
\end{equation}
This function is jointly geodesically convex, i.e., it has GConvex curvature in both $X$ and $Y$ and is GIncreasing. It is non-convex in the Euclidean setting.

\paragraph{Riemannian Metric.} $\text{distance}(X,Y)$ returns the distance with respect to the affine-invariant metric:
\begin{equation}
\text{distance}(X,Y) := \|{\log(Y^{-1/2} X Y^{-1/2})}\|_F.
\end{equation}
It is GConvex and is neither GIncreasing nor GDecreasing, hence its monotonicity is unknown, i.e., GAnyMono.

\paragraph{Quadratic Form.} Fix $h \in \mathbb{R}^d$. The function $\text{quad\_form}(h, X) = h^T X h$ is g-convex and GIncreasing. It is also convex in the Euclidean setting.

\paragraph{Spectral Radius.} The function
\begin{equation}
\text{eigmax}(X) := \sup_{\|y\|_2=1} y^T X y,
\end{equation}
which returns the maximum eigenvalue of $X$, is g-convex and GIncreasing. It is also convex in the Euclidean setting.

\paragraph{Log Quadratic Form.} Let $h_i \in \mathbb{R}^d$ be nonzero vectors for $i = 1, \ldots, n$. Then
\begin{equation}
\text{log\_quad\_form}(\{h_1, \ldots, h_n\}, X) = \log\left(\sum_{i=1}^n h_i^T X^r h_i\right), \quad r \in \{-1, 1\}.
\end{equation}
This is a g-convex function and GIncreasing. It is non-convex in the Euclidean setting.

\paragraph{Symmetric Gauge Functions.} Symmetric gauge functions provide a rich class of geodesically convex spectral functions:

\begin{definition}[Symmetric Gauge Functions]
A map $\Phi: \mathbb{R}^d \rightarrow \mathbb{R}_+$ is called a symmetric gauge function if
\begin{enumerate}
\item $\Phi$ is a norm
\item $\Phi(Px) = \Phi(x)$ for all $x \in \mathbb{R}^n$ and all $n \times n$ permutation matrices $P$ (symmetric property)
\item $\Phi(\alpha_1 x_1, \ldots, \alpha_n x_n) = \Phi(x_1, \ldots, x_n)$ for all $x \in \mathbb{R}^n$ and $\alpha_k \in \{±1\}$ (gauge invariant or absolute property)
\end{enumerate}
\end{definition}

\begin{proposition}[Symmetric Gauge Functions are g-convex]
Let $\Phi: \mathbb{R}^d \rightarrow \mathbb{R}$ be a symmetric gauge function. Then the function $f(A) := \Phi(\lambda(A))$ is geodesically convex, where $\lambda(A) = \{\lambda_1(A), \ldots, \lambda_d(A)\} \in \mathbb{R}^d$ is the eigenspectrum of $A$.
\end{proposition}

Important examples of symmetric gauge functions include:

\paragraph{Ky Fan Norms.} The $k$-Ky Fan function of $X$ is the sum of the top $k$ eigenvalues:
\begin{equation}
\Phi(X) = \sum_{i=1}^k \lambda_i^{\downarrow}(X), \quad 1 \leq k \leq d,
\end{equation}
where $\lambda_i^{\downarrow}(X)$ is the sorted spectrum of $X$. This is implemented as $\text{eigsummax}(X, k)$.

\paragraph{Schatten Norms.} The $p$-Schatten norm for $p \geq 1$ is defined as
\begin{equation}
\Phi(X) = \left(\sum_{i=1}^d \lambda_i^p(X)\right)^{1/p}.
\end{equation}
This is implemented as $\text{schatten\_norm}(X, p)$.

\subsubsection{Matrix-valued Atoms}

\paragraph{Conjugation.} Let $X \in \mathcal{P}_d$ and $A \in \mathbb{R}^{n \times n}$, then $\text{conjugation}(X, A) = A^T X A$. This atom has GConvex curvature and is GIncreasing. It is affine in the Euclidean setting.

\paragraph{Adjoint.} Let $X \in \mathcal{P}_d$, then $\text{adjoint}(X) = X^T$ has GConvex curvature and is GIncreasing. It is affine in the Euclidean setting.

\paragraph{Inverse.} Let $X \in \mathcal{P}_d$, then $\text{inv}(X) = X^{-1}$ has GConvex curvature and is GDecreasing. It is also convex in the Euclidean setting.

\paragraph{Hadamard Product.} Let $X \in \mathcal{P}_d$, then $\text{hadamard\_product}(X, B) = X \odot B$ has GConvex curvature and is GIncreasing. It is affine in the Euclidean setting.

These atoms, combined with the rules presented earlier, provide a comprehensive framework for verifying geodesic convexity of functions defined on the symmetric positive definite manifold.

\section{Implementation in Julia}

We have implemented the DGCP framework in a Julia package called SymbolicAnalysis.jl, which provides functionality for testing and certifying DGCP-compliant expressions. In this section, we describe the key components of our implementation and demonstrate its use through examples.

\subsection{Software Architecture}

The implementation of DGCP in SymbolicAnalysis.jl is based on the foundation of symbolic computation and rewriting capability provided by the Symbolics.jl package. This approach allows us to represent mathematical expressions as abstract syntax trees, which can then be analyzed and transformed according to the rules of geodesic convexity.

Each expression is represented as a tree, where the nodes represent functions (or atoms), and the leaves represent variables or constants. This representation enables the propagation of function properties, such as curvature and monotonicity, through the expression tree.

Previous implementations of disciplined programming, such as CVXPY and Convex.jl, define a class for each atom. We take a different approach in our DGCP implementation. The relevant properties, such as domain, sign, curvature, and monotonicity, are added as metadata to the expression nodes, and then propagated by looking up the corresponding property for every atomic function. The DGCP-compliant rules are implemented using rule-based term rewriting provided by SymbolicUtils.jl.

For analyzing arbitrary expressions, the properties are recursively added by a post-order tree traversal of the expression tree. This approach allows for greater flexibility and modularity in defining new atoms and rules, enabling the incorporation of domain-specific atoms. Since the atoms are directly the Julia functions, the DGCP implementation avoids the need to create and maintain separate implementations of numerical routines.

\subsection{Integration with Optimization.jl}

SymbolicAnalysis.jl is designed to work seamlessly with the broader Julia optimization ecosystem, particularly Optimization.jl. This integration allows users to verify the geodesic convexity of their optimization problems before solving them, providing valuable insights and potential simplifications.

The integration works through a $structural\_analysis$ keyword argument to the $OptimizationProblem$ constructor in Optimization.jl. When this argument is set to `true`, the system attempts to symbolically trace through the objective and constraints, analyzing their convexity structure.

For Riemannian optimization problems on the symmetric positive definite manifold, this analysis can verify geodesic convexity, providing guarantees about the existence of global minima and the convergence of optimization algorithms.

\subsection{Example: Karcher Mean Computation}

To illustrate the use of SymbolicAnalysis.jl for geodesic convexity verification, let's consider the problem of computing the Karcher mean of a set of symmetric positive definite matrices.

Given a set of symmetric positive definite matrices $\{A_j\} \subseteq \mathcal{P}_d$, the Karcher mean is defined as the solution to the problem
\begin{equation}
X^* = argmin_{X \succ 0} \left[ \phi(X) = \sum_{i=1}^m w_i \delta_R^2(X, A_i) \right],
\end{equation}
where $w_i \geq 0$ are weights and $\delta_R$ is the Riemannian distance function.

The Karcher mean has applications in medical imaging, kernel methods, and interpolation. It is a geodesically convex problem, but not Euclidean convex. Furthermore, for $m > 2$, it does not admit a closed-form solution, requiring iterative optimization methods.

Using SymbolicAnalysis.jl, we can verify the geodesic convexity of this problem:

\begin{verbatim}
using SymbolicAnalysis, Manifolds, LinearAlgebra

# Define the manifold and data
M = SymmetricPositiveDefinite(5)
m = 5
q = Matrix{Float64}(I, 5, 5) .+ 2.0
data = [exp(M, q, 0.005 * rand(M; vector_at = q)) for i in 1:m]

# Define the objective function
f(x) = sum(distance(M, x, data[i])^2 for i in 1:m)

# Analyze the geodesic convexity
analyze_res = analyze(f, M)
println(analyze_res.gcurvature)  # Output: GConvex
\end{verbatim}

The analysis confirms that the problem is geodesically convex, validating the theoretical result and providing assurance about the convergence of Riemannian optimization algorithms.

\subsection{Example: Matrix Square Root}

Computing the square root $A^{1/2}$ of a symmetric positive definite matrix $A \in \mathcal{P}_d$ is an important subroutine in many applications. Various optimization approaches have been proposed for this problem, including a geodesically convex formulation by Sra:
\begin{equation}
\min_{X \in \mathcal{P}_d} \phi(X) := \delta_S^2(X, A) + \delta_S^2(X, I),
\end{equation}
where $\delta_S$ is the S-divergence.

We can verify the geodesic convexity of this formulation using SymbolicAnalysis.jl:

\begin{verbatim}
using SymbolicAnalysis, Manifolds, LinearAlgebra

# Define the manifold and data
M = SymmetricPositiveDefinite(5)
A = randn(5, 5)
A = A * A'  # Make it SPD

# Define the objective function
function f(X)
    return sdivergence(X, A) + sdivergence(X, Matrix{Float64}(I, 5, 5))
end

# Analyze the geodesic convexity
analyze_res = analyze(f, M)
println(analyze_res.gcurvature)  # Output: GConvex
\end{verbatim}

Again, the analysis confirms the geodesic convexity of the problem, validating the theoretical result.

\section{Applications of DGCP}

In this section, we demonstrate the application of DGCP to several important problems in statistics, machine learning, and optimization. For each application, we formulate the problem, verify its geodesic convexity using our framework, and discuss the implications for optimization.

\subsection{Robust Covariance Estimation}

Estimating covariance matrices is a fundamental task in statistics and machine learning. In many applications, robustness to outliers is essential. Tyler's M-estimator and related estimators provide robust alternatives to the sample covariance matrix.

Tyler's M-estimator can be formulated as the solution to the following optimization problem:
\begin{equation}
\min_{\Sigma \in \mathcal{P}_d} \frac{1}{n} \sum_{i=1}^n \log(x_i^T \Sigma^{-1} x_i) + \frac{1}{d} \log \det(\Sigma),
\end{equation}
where $\{x_i\}_{i=1}^n \subseteq \mathbb{R}^d$ are the observed data points.

While this problem appears non-convex in the Euclidean sense, it is geodesically convex. Using DGCP, we can verify this property:

\begin{verbatim}
using SymbolicAnalysis, LinearAlgebra

function tyler_objective(Sigma, xs)
    d = size(Sigma, 1)
    return sum(log_quad_form(x, inv(Sigma)) for x in xs) + (1/d) * logdet(Sigma)
end

function verify_tyler_convexity()
    # Generate some sample data
    d = 3
    n = 10
    xs = [rand(d) for _ in 1:n]
    
    # Define the manifold
    M = SymmetricPositiveDefinite(d)
    
    # Define symbolic variables for the covariance matrix
    @variables Sigma[1:d, 1:d]
    
    # Define the objective function
    obj = tyler_objective(Sigma, xs)
    
    # Analyze geodesic convexity
    analyze_res = analyze(obj, M)
    return analyze_res.gcurvature
end

# Verify geodesic convexity
result = verify_tyler_convexity()  # Output: GConvex
\end{verbatim}

The geodesic convexity of Tyler's M-estimator has important implications for optimization. It guarantees that any local minimum is a global minimum, and that Riemannian optimization algorithms will converge to the globally optimal solution regardless of initialization.

\subsection{Brascamp-Lieb Constants}

The Brascamp-Lieb inequalities form an important class of inequalities in functional analysis and probability theory. They encompass many well-known inequalities, such as Hölder's inequality and the Loomis-Whitney inequality. Beyond their applications in mathematics, Brascamp-Lieb inequalities have been used in machine learning and information theory.

The computation of Brascamp-Lieb constants can be formulated as an optimization problem on the positive definite matrices:
\begin{equation}
\min_{X \in \mathcal{P}_d} \left[ F(X) = -\log \det(X) + \sum_i w_i \log \det(A_i^T X A_i) \right],
\end{equation}
where $A_i$ are given matrices and $w_i$ are weights.

This problem is geodesically convex but not Euclidean convex. We can verify its geodesic convexity using DGCP:

\begin{verbatim}
using SymbolicAnalysis, LinearAlgebra

function brascamp_lieb_objective(X, As, ws)
    obj = -logdet(X)
    for (A, w) in zip(As, ws)
        obj += w * logdet(conjugation(X, A))
    end
    return obj
end

function verify_bl_convexity()
    # Define problem dimensions
    d = 3
    m = 2
    
    # Generate random data
    As = [randn(d, d) for _ in 1:m]
    ws = rand(m)
    ws ./= sum(ws)  # Normalize weights
    
    # Define the manifold
    M = SymmetricPositiveDefinite(d)
    
    # Define symbolic variables
    @variables X[1:d, 1:d]
    
    # Define the objective function
    obj = brascamp_lieb_objective(X, As, ws)
    
    # Analyze geodesic convexity
    analyze_res = analyze(obj, M)
    return analyze_res.gcurvature
end

# Verify geodesic convexity
result = verify_bl_convexity()  # Output: GConvex
\end{verbatim}

The geodesic convexity of this problem has led to the development of Riemannian optimization approaches for computing Brascamp-Lieb constants, which have proven more efficient than traditional methods.

\section{Conclusion}

In this chapter, we have introduced Disciplined Geodesically Convex Programming (DGCP), a framework for verifying and exploiting geodesic convexity on Riemannian manifolds. We have presented the theoretical foundations of DGCP, including rules for preserving geodesic convexity under various operations, with a particular focus on the manifold of symmetric positive definite matrices.

We have implemented this framework in the Julia package SymbolicAnalysis.jl, which provides a user-friendly interface for verifying the geodesic convexity of optimization problems. Through examples and applications, we have demonstrated the practical utility of this framework for problems in statistics, machine learning, and optimization.

The key contributions of DGCP include:

\begin{enumerate}
\item A systematic framework for verifying geodesic convexity, extending the concept of disciplined convex programming to Riemannian manifolds
\item A comprehensive set of rules and atoms for the manifold of symmetric positive definite matrices, capturing many important operations in practical applications
\item An efficient implementation in Julia that integrates with the broader optimization ecosystem
\item Demonstrations of the framework's application to important problems, including robust covariance estimation, Brascamp-Lieb constant computation, and diffusion tensor imaging
\end{enumerate}

DGCP opens up new possibilities for optimization on manifolds, allowing practitioners to leverage the theoretical guarantees of geodesic convexity for a wider range of problems. By automating the verification of geodesic convexity, it reduces the barrier to entry for manifold optimization and promotes the adoption of geometrically informed approaches to non-convex problems.

As manifold optimization continues to gain prominence in machine learning, statistics, and other fields, frameworks like DGCP will play an increasingly important role in advancing both the theory and practice of optimization on non-Euclidean domains.
%\chapter{Software Design and Implementation}

This chapter presents the software design and implementation of the optimization methods developed in this thesis. We begin by discussing the design principles that guided our implementation, emphasizing composability, extensibility, and performance. We then describe the architecture of Optimization.jl, the core framework that unifies our approach to optimization across different problem domains. We detail the implementation of specific components, including the disciplined geodesically convex programming framework, the GPU-accelerated particle swarm optimization, and the augmented Lagrangian method with stochastic inner optimizers. Finally, we discuss the integration of these components with the broader scientific computing ecosystem in Julia, highlighting how our software enables researchers across disciplines to apply advanced optimization techniques to their specific problems.

\section{Design Principles}

The development of scientific software requires careful consideration of various, often competing, factors: ease of use, computational efficiency, extensibility, and correctness. In developing the software components of this thesis, we adhered to the following design principles:

\subsection{Composability}

Scientific computing workflows often involve multiple stages and components. For example, an optimization problem might require automatic differentiation for gradient computation, specialized solvers for the specific problem structure, and visualization tools for analyzing the results. A composable software design allows these components to work together seamlessly, enabling users to mix and match different approaches based on their specific needs.

Our implementation leverages Julia's multiple dispatch system and well-defined interfaces to achieve composability. Key interfaces, such as those for problem definition, automatic differentiation, and solver selection, are designed to be interoperable, allowing users to combine different components without friction.

\subsection{Extensibility}

The field of optimization is constantly evolving, with new algorithms, problem formulations, and application domains emerging regularly. A well-designed software framework should accommodate these developments without requiring a complete redesign. Our implementation emphasizes extensibility through:

\begin{enumerate}
\item \textbf{Abstract interfaces}: Core concepts like optimization problems, optimizers, and automatic differentiation backends are defined through abstract interfaces, allowing new implementations to be added without modifying existing code.

\item \textbf{Plugin architecture}: New optimizers and problem types can be added through a plugin system, enabling the ecosystem to grow independently of the core framework.

\item \textbf{Minimal assumptions}: The framework makes minimal assumptions about the structure of optimization problems, accommodating a wide range of problem types and domains.
\end{enumerate}

\subsection{Performance}

Scientific computing often involves computationally intensive tasks, making performance a critical consideration. Our implementation prioritizes performance through:

\begin{enumerate}
\item \textbf{Zero-overhead abstractions}: Julia's type system and compilation model allow us to provide high-level abstractions without sacrificing performance.

\item \textbf{Specialized implementations}: For performance-critical components, we provide specialized implementations that leverage problem-specific structure.

\item \textbf{Hardware acceleration}: Where appropriate, we leverage GPU computing and other forms of hardware acceleration to improve performance.
\end{enumerate}

\subsection{User Experience}

Advanced optimization techniques should be accessible to researchers across disciplines, not just experts in optimization theory. Our implementation emphasizes a positive user experience through:

\begin{enumerate}
\item \textbf{Consistent interfaces}: Common operations, such as defining problems, selecting solvers, and analyzing results, follow consistent patterns across different problem types.

\item \textbf{Informative feedback}: The software provides clear error messages and warnings, helping users diagnose and resolve issues.

\item \textbf{Comprehensive documentation}: Each component is thoroughly documented, with examples and tutorials covering common use cases.
\end{enumerate}

\section{Optimization.jl: A Unified Optimization Framework}

At the core of our software implementation is Optimization.jl, a unified framework for optimization in Julia. This framework provides a common interface to over 25 optimization libraries, encompassing more than 100 optimization solvers across various classes, from global to local, from convex to non-convex, and from unconstrained to constrained optimization.

\subsection{Overview and Architecture}

Optimization.jl follows a modular architecture organized around several key abstractions:

\begin{enumerate}
\item \textbf{OptimizationProblem}: Represents an optimization problem, including the objective function, constraints, initial guess, and problem parameters.

\item \textbf{OptimizationFunction}: Encapsulates the objective function and its derivatives, which can be provided explicitly or generated automatically using automatic differentiation.

\item \textbf{AbstractOptimizationAlgorithm}: The base type for all optimization algorithms, defining the interface that all solvers must implement.

\item \textbf{OptimizationSolution}: Contains the result of an optimization process, including the solution, objective value, convergence status, and other metadata.
\end{enumerate}

These abstractions form the foundation of the framework, enabling consistent interfaces across different problem types and solvers.

\begin{figure}
\centering
\includegraphics[width=\textwidth]{optimization_architecture.png}
\caption{Architecture of Optimization.jl, showing the relationships between key components and their integration with external libraries.}
\label{fig:architecture}
\end{figure}

The framework employs a plugin system to integrate with external optimization libraries. Each supported library has a corresponding wrapper package (e.g., OptimizationOptimJL.jl, OptimizationNLopt.jl) that adapts the library's interface to the Optimization.jl abstractions. This approach allows the framework to grow independently of the core package, with new libraries and solvers being added by the community as needed.

\subsection{Problem Definition and Automatic Differentiation}

A key feature of Optimization.jl is its flexible approach to problem definition and automatic differentiation. Users can define optimization problems using standard Julia functions and leverage various automatic differentiation backends for gradient computation.

The framework supports multiple automatic differentiation backends through the ADTypes.jl package, which provides a consistent interface to different AD systems:

\begin{enumerate}
\item \textbf{ForwardDiff.jl}: Forward-mode automatic differentiation, suitable for problems with few parameters but many outputs.

\item \textbf{ReverseDiff.jl}: Reverse-mode automatic differentiation, suitable for problems with many parameters but few outputs.

\item \textbf{Zygote.jl}: Source-to-source reverse-mode automatic differentiation, optimized for machine learning workloads.

\item \textbf{Enzyme.jl}: LLVM-based automatic differentiation, offering high performance for low-level code.

\item \textbf{FiniteDiff.jl}: Finite difference approximations, suitable when automatic differentiation is not applicable.

\item \textbf{ModelingToolkit.jl}: Symbolic differentiation, enabling symbolic manipulation and optimization of expressions before numerical evaluation.
\end{enumerate}

This flexibility allows users to choose the most appropriate differentiation method for their specific problem, balancing accuracy, performance, and compatibility with existing code.

\subsection{Constraint Handling}

Optimization.jl provides a flexible system for defining and handling constraints. Constraints can be specified as:

\begin{enumerate}
\item \textbf{Box constraints}: Lower and upper bounds on individual variables, specified using the `lb` and `ub` parameters of the OptimizationProblem constructor.

\item \textbf{Equality constraints}: Constraints of the form $h(x) = 0$, specified using the `cons` parameter of the OptimizationFunction constructor and setting equal values in the `lcons` and `ucons` parameters of the OptimizationProblem constructor.

\item \textbf{Inequality constraints}: Constraints of the form $g(x) \leq 0$, specified using the `cons` parameter of the OptimizationFunction constructor and setting appropriate values in the `lcons` and `ucons` parameters of the OptimizationProblem constructor.
\end{enumerate}

The framework automatically routes these constraints to the appropriate solver mechanism, whether that's a dedicated constrained optimization solver or a constraint handling technique like the augmented Lagrangian method.

\subsection{Solver Selection and Configuration}

Optimization.jl provides a unified interface for selecting and configuring optimization solvers. Users can choose from a wide range of solvers across different categories:

\begin{enumerate}
\item \textbf{Local vs. Global}: From local methods like BFGS and Newton to global methods like simulated annealing and genetic algorithms.

\item \textbf{First-order vs. Second-order}: From gradient-based methods like gradient descent to Hessian-based methods like Newton's method.

\item \textbf{Deterministic vs. Stochastic}: From deterministic methods like L-BFGS to stochastic methods like stochastic gradient descent and Adam.

\item \textbf{Unconstrained vs. Constrained}: From unconstrained methods like BFGS to constrained methods like interior point and augmented Lagrangian.
\end{enumerate}

Solver-specific options can be passed as keyword arguments to the `solve` function, enabling fine-grained control over the optimization process. Common options, such as convergence tolerances and maximum iterations, are standardized across solvers for consistency.

\section{Implementation of Disciplined Geodesically Convex Programming}

The disciplined geodesically convex programming framework, described in Chapter 3, is implemented in the SymbolicAnalysis.jl package. This package provides tools for analyzing the geodesic convexity of functions defined on Riemannian manifolds, with a focus on the symmetric positive definite manifold.

\subsection{Symbolic Computation and Rule-Based Analysis}

SymbolicAnalysis.jl leverages the symbolic computation capabilities of the Symbolics.jl package to represent mathematical expressions as abstract syntax trees. These trees can then be analyzed and manipulated using rule-based transformations.

The core of the package is a set of rules for propagating convexity information through expression trees. Each node in the tree, representing an operation or function, is associated with rules that determine how convexity properties propagate from the node's children to the node itself. For example, the sum of two geodesically convex functions is geodesically convex, so the corresponding rule propagates this property accordingly.

\begin{figure}
\centering
\includegraphics[width=0.8\textwidth]{symbolic_analysis.png}
\caption{Example of an expression tree in SymbolicAnalysis.jl, showing the propagation of geodesic convexity properties from leaves to the root.}
\label{fig:symbolic_analysis}
\end{figure}

\subsection{Atom Library for Symmetric Positive Definite Matrices}

The package includes a comprehensive library of "atoms"—basic functions with known geodesic convexity properties—for the symmetric positive definite manifold. These atoms, described in detail in Chapter 3, include:

\begin{enumerate}
\item \textbf{Scalar-valued atoms}: Functions like log-determinant, trace, quadratic forms, and spectral functions.

\item \textbf{Matrix-valued atoms}: Operations like matrix inversion, conjugation, and Hadamard product.
\end{enumerate}

Each atom is implemented as a standard Julia function, with its geodesic convexity properties registered in a central database. This approach allows users to work with familiar mathematical notation while enabling the system to track convexity properties behind the scenes.

\subsection{Integration with Optimization.jl}

SymbolicAnalysis.jl integrates with Optimization.jl through the `structural_analysis` feature. When users create an OptimizationProblem with the `structural_analysis = true` option, the system attempts to symbolically trace through the objective and constraint functions, analyzing their convexity structure.

For Riemannian optimization problems on the symmetric positive definite manifold, this analysis can verify geodesic convexity, providing guarantees about the global optimality of solutions. This integration enables users to leverage the theoretical insights from disciplined geodesically convex programming within the broader Optimization.jl ecosystem.

\section{Implementation of GPU-Accelerated PSO-L-BFGS}

The GPU-accelerated particle swarm optimization with L-BFGS refinement, described in Chapter 4, is implemented in the PSOGPU.jl package. This package provides efficient implementations of various PSO variants, with optional L-BFGS refinement, all accelerated using GPU computing.

\subsection{GPU Kernel Design}

PSOGPU.jl leverages the KernelAbstractions.jl package for writing GPU kernels that are portable across different GPU backends, including NVIDIA CUDA and AMD ROCm. The key kernels implemented in the package include:

\begin{enumerate}
\item \textbf{Position and velocity update kernel}: Updates the positions and velocities of all particles in parallel.

\item \textbf{Objective evaluation kernel}: Evaluates the objective function for all particles in parallel.

\item \textbf{Global best update kernel}: Updates the global best position using parallel reduction.
\end{enumerate}

These kernels are carefully designed to minimize memory transfers between host and device, maximizing the efficiency of the GPU implementation.

\subsection{Synchronous and Asynchronous Variants}

The package implements both synchronous and asynchronous variants of PSO, each with its own trade-offs:

\begin{enumerate}
\item \textbf{ParallelSyncPSOKernel}: The basic synchronous PSO implementation, where all particles are updated using the same global best information.

\item \textbf{ParallelPSOKernel(global\_update = true)}: A more efficient synchronous implementation using a queue-lock mechanism for updating the global best.

\item \textbf{ParallelPSOKernel(global\_update = false)}: An asynchronous implementation where particles evolve independently, reducing synchronization overhead at the cost of exploration quality.
\end{enumerate}

Users can choose the most appropriate variant based on their specific problem characteristics and hardware constraints.

\subsection{L-BFGS Refinement}

The HybridPSO algorithm extends the basic PSO implementation with L-BFGS refinement. This hybrid approach leverages the global exploration capabilities of PSO and the efficient local refinement of L-BFGS.

The L-BFGS implementation is optimized for GPU computing, with careful attention to memory management and kernel design. The implementation supports both forward-mode and reverse-mode automatic differentiation for gradient computation, allowing users to choose the most appropriate method for their problem.

\subsection{Integration with Optimization.jl}

PSOGPU.jl integrates with Optimization.jl through the standard optimizer interface. Users can select and configure PSO variants using the same patterns as any other Optimization.jl solver:

\begin{verbatim}
using Optimization, PSOGPU

# Define the optimization problem
problem = OptimizationProblem(f, x0, p, lb = lower_bounds, ub = upper_bounds)

# Solve using HybridPSO with L-BFGS refinement
solution = solve(problem, HybridPSO(refinement = :LBFGS, 
                                    particle_count = 1000, 
                                    refinement_iterations = 100))
\end{verbatim}

This integration allows users to seamlessly incorporate GPU-accelerated PSO into their existing Optimization.jl workflows.

\section{Implementation of the Stochastic Augmented Lagrangian Method}

The stochastic augmented Lagrangian method, described in Chapter 5, is implemented as part of Optimization.jl itself. This implementation combines the classical augmented Lagrangian method with stochastic optimizers like Adam for the inner minimization problem.

\subsection{Augmented Lagrangian Framework}

The core of the implementation is an augmented Lagrangian framework that handles both equality and inequality constraints. The framework follows the standard augmented Lagrangian approach:

\begin{enumerate}
\item \textbf{Inner minimization}: Minimize the augmented Lagrangian with respect to the optimization variables, keeping the Lagrange multipliers and penalty parameter fixed.

\item \textbf{Lagrange multiplier update}: Update the Lagrange multipliers based on the constraint violations.

\item \textbf{Penalty parameter update}: Increase the penalty parameter if the constraint violations have not decreased sufficiently.
\end{enumerate}

\subsection{Stochastic Inner Optimizers}

The key innovation in our implementation is the use of stochastic optimizers for the inner minimization problem. The implementation supports various stochastic optimizers from the Optimisers.jl package, including:

\begin{enumerate}
\item \textbf{SGD}: Simple stochastic gradient descent with a fixed learning rate.

\item \textbf{Momentum}: SGD with momentum for accelerated convergence.

\item \textbf{Adam}: Adaptive moment estimation, combining momentum and adaptive learning rates.

\item \textbf{RMSProp}: Root mean square propagation, which adapts learning rates based on recent gradient magnitudes.
\end{enumerate}

These optimizers are integrated into the augmented Lagrangian framework through a custom inner optimization loop that handles batched training data and stochastic gradient estimates.

\subsection{Mini-Batch Training Support}

The implementation includes special support for mini-batch training, which is essential for efficiently handling large datasets. This support includes:

\begin{enumerate}
\item \textbf{DataLoader integration}: The OptimizationProblem constructor accepts a MLUtils.DataLoader object, which provides batched access to training data.

\item \textbf{Epoch-based training}: The solve function accepts an epochs parameter, specifying the number of passes through the training data.

\item \textbf{Progress tracking}: The implementation tracks progress at both the batch and epoch levels, providing informative feedback during training.
\end{enumerate}

\subsection{Constraint Handling}

The implementation provides flexible constraint handling through the OptimizationFunction and OptimizationProblem constructors. Users can specify:

\begin{enumerate}
\item \textbf{Equality constraints}: Constraints of the form $h(x) = 0$, specified by setting equal values in lcons and ucons.

\item \textbf{Inequality constraints}: Constraints of the form $g(x) \leq 0$, specified by setting appropriate values in lcons and ucons.

\item \textbf{Constraint gradients}: Gradients of constraints can be provided explicitly or generated automatically using the chosen automatic differentiation backend.
\end{enumerate}

This flexible constraint specification allows users to express a wide range of constrained optimization problems within the Optimization.jl framework.

\section{Integration with the Julia Ecosystem}

One of the key strengths of our software implementation is its integration with the broader Julia ecosystem for scientific computing. This integration enables users to combine our optimization tools with other packages for data analysis, numerical computing, and visualization.

\subsection{Automatic Differentiation Ecosystem}

Optimization.jl leverages Julia's rich ecosystem of automatic differentiation tools through a consistent interface defined in ADTypes.jl. This allows users to:

\begin{enumerate}
\item \textbf{Choose the most appropriate AD backend}: Different AD systems have different strengths and trade-offs. ForwardDiff.jl is often faster for problems with few variables, while Zygote.jl is typically more efficient for neural networks.

\item \textbf{Mix AD with hand-coded derivatives}: Users can provide explicit gradients for some parts of their objective function while letting the AD system handle the rest.

\item \textbf{Leverage library-specific AD optimizations}: Many Julia libraries provide custom AD rules that improve the efficiency of differentiation for their specific operations.
\end{enumerate}

\subsection{Differential Equations Ecosystem}

The integration with DifferentialEquations.jl, a powerful suite for solving differential equations, is particularly important for scientific applications. This integration allows users to:

\begin{enumerate}
\item \textbf{Optimize parameters of differential equation models}: Users can define objective functions that involve solving differential equations, and Optimization.jl will handle the interface with the differential equation solver.

\item \textbf{Train neural ODEs and other SciML models}: The integration enables efficient training of scientific machine learning models that combine differential equations with neural networks.

\item \textbf{Leverage sensitivity analysis techniques}: The SciMLSensitivity.jl package provides efficient methods for computing gradients of differential equation solutions, which Optimization.jl can use for optimization.
\end{enumerate}

\begin{figure}
\centering
\includegraphics[width=\textwidth]{ecosystem_integration.png}
\caption{Integration of Optimization.jl with the broader Julia ecosystem for scientific computing, showing key connections with automatic differentiation, differential equations, and other domains.}
\label{fig:ecosystem}
\end{figure}

\subsection{Machine Learning Ecosystem}

The integration with Julia's machine learning ecosystem enables applications in deep learning and statistical modeling:

\begin{enumerate}
\item \textbf{Training neural networks}: Optimization.jl can be used to train neural networks defined using packages like Flux.jl and Lux.jl.

\item \textbf{Bayesian inference}: The integration with Turing.jl enables gradient-based inference methods for Bayesian models.

\item \textbf{Reinforcement learning}: Optimization.jl can be used for policy optimization in reinforcement learning applications.
\end{enumerate}

\subsection{Domain-Specific Applications}

The generality and extensibility of Optimization.jl enable its application across diverse scientific domains:

\begin{enumerate}
\item \textbf{Computational physics}: Optimizing parameters in physics models and simulations.

\item \textbf{Computational biology}: Parameter estimation for biological systems and molecular modeling.

\item \textbf{Control systems}: Designing and optimizing controllers for dynamic systems.

\item \textbf{Signal processing}: Optimizing filters and other signal processing components.

\item \textbf{Operations research}: Solving scheduling, routing, and resource allocation problems.
\end{enumerate}

This breadth of applications demonstrates the impact of a well-designed, composable optimization framework for scientific computing.

\section{Documentation and Examples}

A critical aspect of scientific software is comprehensive documentation and examples that help users apply the tools to their specific problems. Our documentation strategy includes:

\subsection{API Documentation}

Each component of the software is documented at the API level, with clear descriptions of functions, types, and their parameters. The documentation is automatically generated from docstrings in the code, ensuring that it stays in sync with the implementation.

\subsection{Tutorials and Examples}

The documentation includes a range of tutorials and examples, covering common use cases and illustrating best practices. These include:

\begin{enumerate}
\item \textbf{Basic usage}: Getting started with Optimization.jl, defining problems, and selecting solvers.

\item \textbf{Advanced topics}: Working with constraints, custom gradient definitions, and specialized problem types.

\item \textbf{Domain-specific examples}: Applications in machine learning, differential equations, control systems, and other domains.
\end{enumerate}

\subsection{Performance Tips}

The documentation provides guidance on optimizing performance, including:

\begin{enumerate}
\item \textbf{Choosing the right AD backend}: Guidelines for selecting the most appropriate automatic differentiation method based on problem characteristics.

\item \textbf{Exploiting problem structure}: Techniques for leveraging sparsity, symmetry, and other structural properties to improve performance.

\item \textbf{Hardware acceleration}: Tips for effectively using GPU computing and other forms of hardware acceleration.
\end{enumerate}

\subsection{Community Resources}

Beyond the official documentation, the software is supported by a range of community resources:

\begin{enumerate}
\item \textbf{Forum discussions}: The Julia Discourse forum includes a section dedicated to optimization, where users can ask questions and share experiences.

\item \textbf{Chat channels}: Real-time support is available through the Julia Slack and Zulip chat platforms.

\item \textbf{Video tutorials}: Community members have created video tutorials covering various aspects of the software.
\end{enumerate}

These resources form a supportive ecosystem that helps users overcome challenges and make the most of the software.

\section{Case Studies}

To illustrate the practical impact of our software implementation, we present several case studies drawn from real-world applications across different domains.

\subsection{Training a Physics-Informed Neural Network}

Physics-informed neural networks (PINNs) combine the approximation power of neural networks with physical constraints derived from governing equations. Training PINNs requires minimizing a loss function that includes both data fitting terms and physics residual terms:

\begin{equation}
\mathcal{L} = \mathcal{L}_{data} + \lambda \mathcal{L}_{physics}
\end{equation}

where $\mathcal{L}_{data}$ measures the discrepancy between network predictions and observed data, $\mathcal{L}_{physics}$ measures the violation of physical laws, and $\lambda$ is a weight parameter.

Using Optimization.jl, we can define this problem and solve it using a variety of optimizers. The example below shows how to train a PINN for solving a partial differential equation using the ADAM optimizer:

\begin{verbatim}
using Optimization, OptimizationOptimisers, Lux

# Define the neural network
model = Chain(Dense(1, 32, tanh), Dense(32, 1))

# Define the loss function, including data and physics components
function loss(θ, p)
    x_data, y_data, x_physics = p
    
    # Data loss
    y_pred = model(x_data, θ)[1]
    data_loss = mean((y_pred - y_data).^2)
    
    # Physics loss
    u = x -> model([x], θ)[1][1]
    du_dx = x -> ForwardDiff.derivative(u, x)
    d2u_dx2 = x -> ForwardDiff.derivative(du_dx, x)
    
    physics_loss = mean(d2u_dx2.(x_physics) .+ u.(x_physics))
    
    # Total loss
    return data_loss + 0.1 * physics_loss
end

# Create optimization problem
optf = OptimizationFunction(loss, Optimization.AutoZygote())
prob = OptimizationProblem(optf, parameters, data)

# Solve using ADAM
sol = solve(prob, Adam(0.01), maxiters=1000)
\end{verbatim}

This example demonstrates how Optimization.jl simplifies the integration of neural networks, automatic differentiation, and physical constraints, enabling efficient training of PINNs for scientific applications.

\subsection{Parameter Estimation in Epidemiological Models}

Epidemiological models, such as the SIR (Susceptible-Infected-Recovered) model, are commonly used to understand the spread of infectious diseases. Estimating the parameters of these models from observed data is a challenging optimization problem due to the nonlinear dynamics and constraints on the parameters.

Using Optimization.jl with the stochastic augmented Lagrangian method, we can solve this parameter estimation problem while respecting biological constraints:

\begin{verbatim}
using Optimization, DifferentialEquations

# Define the SIR model
function sir_model!(du, u, p, t)
    S, I, R = u
    β, γ = p
    
    du[1] = -β * S * I
    du[2] = β * S * I - γ * I
    du[3] = γ * I
end

# Define the loss function
function loss(p, data)
    β, γ = p
    tspan = (0.0, 60.0)
    u0 = [0.99, 0.01, 0.0]
    prob = ODEProblem(sir_model!, u0, tspan, p)
    sol = solve(prob, Tsit5(), saveat=data.times)
    
    # Compare model predictions with observed data
    return sum(abs2, sol - data.observations)
end

# Define constraints
function constraints(res, p, data)
    β, γ = p
    
    # Constraints: β > 0, γ > 0, R₀ = β/γ < 2.5
    res[1] = p[1]        # β
    res[2] = p[2]        # γ
    res[3] = p[1] / p[2] # R₀
end

# Create optimization problem
optf = OptimizationFunction(loss, Optimization.AutoForwardDiff(), cons=constraints)
prob = OptimizationProblem(optf, [0.5, 0.1], data, lcons=[0.0, 0.0, 0.0], ucons=[Inf, Inf, 2.5])

# Solve using stochastic augmented Lagrangian method
sol = solve(prob, SALM(inner_optimizer=Adam()), maxiters=100)
\end{verbatim}

This example showcases how Optimization.jl seamlessly integrates with differential equation solvers, automatic differentiation, and constraint handling, enabling sophisticated parameter estimation for complex dynamical systems.

\subsection{Robust Covariance Estimation on the SPD Manifold}

Estimating covariance matrices that are robust to outliers is important in many statistical applications. Tyler's M-estimator is a popular choice for robust covariance estimation, but the optimization problem is non-convex in the Euclidean sense, requiring specialized algorithms.

Using Optimization.jl with the disciplined geodesically convex programming framework, we can exploit the geodesic convexity of this problem on the manifold of symmetric positive definite matrices:

\begin{verbatim}
using Optimization, OptimizationManopt, SymbolicAnalysis, Manifolds

# Define the manifold
M = SymmetricPositiveDefinite(d)

# Define Tyler's M-estimator objective
function tyler_objective(Σ, data)
    d = size(Σ, 1)
    n = length(data)
    
    inv_Σ = inv(Σ)
    
    obj = 0.0
    for x in data
        obj += log(dot(x, inv_Σ * x))
    end
    
    obj = obj / n + log(det(Σ)) / d
    return obj
end

# Check geodesic convexity
analysis = analyze(tyler_objective, M)
println("Geodesic convexity: ", analysis.gcurvature)

# Create optimization problem
optf = OptimizationFunction(tyler_objective, Optimization.AutoZygote())
prob = OptimizationProblem(optf, initial_guess, data, manifold=M)

# Solve using Riemannian optimization
sol = solve(prob, GradientDescentOptimizer())
\end{verbatim}

This example demonstrates how the integration of SymbolicAnalysis.jl with Optimization.jl enables the verification and exploitation of geodesic convexity for optimization on Riemannian manifolds.

\section{Conclusion}

This chapter has presented the software design and implementation of the optimization methods developed in this thesis. The implementation is centered around Optimization.jl, a unified framework for optimization in Julia, with specialized packages for disciplined geodesically convex programming, GPU-accelerated particle swarm optimization, and stochastic augmented Lagrangian methods.

The design principles of composability, extensibility, performance, and user experience have guided our implementation, resulting in software that is both powerful and accessible. The integration with the broader Julia ecosystem for scientific computing enables applications across diverse domains, from machine learning to differential equations and beyond.

The case studies presented demonstrate the practical impact of our software, showing how it enables researchers to solve complex optimization problems with minimal overhead. The comprehensive documentation and community resources ensure that users can effectively apply these tools to their specific problems.

In the next chapter, we evaluate the performance of our methods on a range of benchmark problems and real-world applications, demonstrating the practical advantages of our approaches over existing methods.


%%% Appendicies of thesis  %%%%%%%%%%%%%%%%%%%%%%%%%%%%%%%%%%%%%%%%%%%%%%%%%%%%%%%%%%%%%%%%%%%%%%%%

\appendix
% From mitthesis package
% Version: 1.01, 2023/07/04
% Documentation: https://ctan.org/pkg/mitthesis


\chapter{Code listing}

This example uses the \texttt{listings} package.

\bigskip

\lstdefinestyle{mystyle}{
    backgroundcolor=\color{CadetBlue!15!white},   
    commentstyle=\color{Red3},
    numberstyle=\tiny\color{gray},
    stringstyle=\color{Blue3},
    basicstyle=\small\ttfamily,
    breakatwhitespace=false,         
    breaklines=true,                 
    numbers=left,                    
    numbersep=5pt,                  
    showspaces=false,                
    showstringspaces=false,
    showtabs=false,                  
    tabsize=2
}%
\lstset{language=[5.3]Lua,style={mystyle}}%

\begin{lstlisting}
function print_rate(kappa,xMin,xMax,npoints,option)
     local c = 1-kappa*kappa
     local croot = (1-kappa*kappa)^(1/2)
     local logx = math.log(xMin)
     local psi = 0
     
     local xstep = (math.log(xMax)-math.log(xMin))/(npoints-1)
     
     arg0 = math.sqrt(xMin/c)
     psi0 = (1/c)*math.exp((kappa*arg0)^2)*(erfc(kappa*arg0)-erfc(arg0))
     
     if option~=[[]] then
  		 tex.sprint("\\addplot+["..option.."] coordinates{") 
  		 -- addplot+ for color cycle to work
     else
  		 tex.sprint("\\addplot+ coordinates{")
     end
     tex.sprint("("..xMin..","..psi0..")")
     
     for i=1, (npoints-1) do
  		 x = math.exp(logx + xstep)
  		 arg = math.sqrt(x/c)
  		 karg = kappa*arg
  		 if karg<5 then 
		 -- this break compensates for exp(karg^2), which multiplies the error in the erf approximation...
  		    logpsi = -math.log(croot) + karg^2 + math.log(erfc(karg)-erfc(arg))
  		    psi = math.exp(logpsi)
  		 else
  		    psi = (1/(karg) - 1/(2*(karg^3)) + 3/(4*(arg^5)) )/(1.77245385*croot)
  		    -- this is the large x asymptote of the reaction rate
  		 end
  		 logx = math.log(x)
  		 tex.sprint("("..x..","..psi..")")
     end
     tex.sprint("}")
end
\end{luacode*}
\end{lstlisting}

%% MIT Thesis class sample appendix with a long table
%% version 1.02, 2024/09/07

\chapter{One-term coefficients for heat conduction}

\section{A multipage table of numbers}
This example uses the \texttt{longtable} package: $\theta = A_1 f_1 \exp(-\lambda_1^2\mkern2mu\mathrm{Fo})$, $\overline{\theta} = D_1 \exp(-\lambda_1^2\mkern2mu\mathrm{Fo})$.


%% These four lines change the dcolumn to use text figures, instead of math figures.
%% The reason is that some mitthesis font sets use different typefaces for text and math
%% See: https://tex.stackexchange.com/a/376127/119566
\makeatletter
	\newcolumntype{T}[3]{>{\textfont0 =\the\font\DC@{#1}{#2}{#3}}c<{\DC@end}}
	\newcolumntype{d}[1]{T{.}{.}{#1}}% overwrites definition in root .tex file + gives warning message
\makeatother

{\footnotesize
% read documentation of longtable package for info on setting up a long table
% read documentation of array package and dcolumn package for info on column format specifiers
\renewcommand{\doublerulesep}{0pt}%
\newcolumntype{X}{>{\hspace{1ex}}c@{\hspace{2ex}}c@{\hspace{2ex}}c<{\hspace{1ex}}}%

\begin{longtable}{|||d{3.2}|X|X|X|||}

\caption{One-term coefficients for one-dimensional heat conduction with a convective boundary condition. Data follow H. D. Baehr and K. Stephan~\cite{baehr1998}.}%

\\
\hline\hline\hline
&&&&&&&&&\\[-7pt]
& \multicolumn{3}{c|}{\textsf{\textit{Plate}}} & \multicolumn{3}{c|}{\textsf{\textit{Cylinder}}} & \multicolumn{3}{c|||}{\textsf{\textit{Sphere}}}
\\ 
\cline{2-10} 
\multicolumn{1}{|||c|}{\raisebox{1.5ex}[0cm][0cm]{Bi}} 
       & $\lambda_1$\rule[0pt]{0pt}{11pt} & $A_1$ & $D_1$ & $\lambda_1$ & $A_1$ & $D_1$ & $\lambda_1$ & $A_1$ & $D_1$ 
\\  
\hline  
\endfirsthead
\caption[]{(continued)} \\
\hline\hline\hline
&&&&&&&&&\\[-7pt]
& \multicolumn{3}{c|}{\textsf{\textit{Plate}}} & \multicolumn{3}{c|}{\textsf{\textit{Cylinder}}} & \multicolumn{3}{c|||}{\textsf{\textit{Sphere}}}
\\ 
\cline{2-10}
\multicolumn{1}{|||c|}{\raisebox{1.5ex}[0cm][0cm]{Bi}} 
       & $\lambda_1$\rule[0pt]{0pt}{11pt} & $A_1$ & $D_1$ & $\lambda_1$ & $A_1$ & $D_1$ & $\lambda_1$ & $A_1$ & $D_1$ 
\\ \hline
&&&&&&&&&\\[-1ex]
\endhead
\hline\hline\hline
\endfoot
\hline\hline\hline
\endlastfoot 
0.01   & 0.09983  & 1.0017  & 1.0000  & 0.14124   & 1.0025  & 1.0000  & 0.17303  & 1.0030  & 1.0000\rule[0pt]{0pt}{15pt} \\ 
0.02   & 0.14095  & 1.0033  & 1.0000  & 0.19950   & 1.0050  & 1.0000  & 0.24446  & 1.0060  & 1.0000 \\ 
0.03   & 0.17234  & 1.0049  & 1.0000  & 0.24403   & 1.0075  & 1.0000  & 0.29910  & 1.0090  & 1.0000 \\ 
0.04   & 0.19868  & 1.0066  & 1.0000  & 0.28143   & 1.0099  & 1.0000  & 0.34503  & 1.0120  & 1.0000 \\  
0.05   & 0.22176  & 1.0082  & 0.9999  & 0.31426   & 1.0124  & 0.9999  & 0.38537  & 1.0150  & 1.0000 \\  
0.06   & 0.24253  & 1.0098  & 0.9999  & 0.34383   & 1.0148  & 0.9999  & 0.42173  & 1.0179  & 0.9999 \\   
0.07   & 0.26153  & 1.0114  & 0.9999  & 0.37092   & 1.0173  & 0.9999  & 0.45506  & 1.0209  & 0.9999 \\  
0.08   & 0.27913  & 1.0130  & 0.9999  & 0.39603   & 1.0197  & 0.9999  & 0.48600  & 1.0239  & 0.9999 \\  
0.09   & 0.29557  & 1.0145  & 0.9998  & 0.41954   & 1.0222  & 0.9998  & 0.51497  & 1.0268  & 0.9999 \\  
0.10   & 0.31105  & 1.0161  & 0.9998  & 0.44168   & 1.0246  & 0.9998  & 0.54228  & 1.0298  & 0.9998 \\[6pt]   
%
0.15   & 0.37788  & 1.0237  & 0.9995  & 0.53761   & 1.0365  & 0.9995  & 0.66086  & 1.0445  & 0.9996 \\*  
0.20   & 0.43284  & 1.0311  & 0.9992  & 0.61697   & 1.0483  & 0.9992  & 0.75931  & 1.0592  & 0.9993 \\  
0.25   & 0.48009  & 1.0382  & 0.9988  & 0.68559   & 1.0598  & 0.9988  & 0.84473  & 1.0737  & 0.9990 \\  
0.30   & 0.52179  & 1.0450  & 0.9983  & 0.74646   & 1.0712  & 0.9983  & 0.92079  & 1.0880  & 0.9985 \\  
0.40   & 0.59324  & 1.0580  & 0.9971  & 0.85158   & 1.0931  & 0.9970  & 1.05279  & 1.1164  & 0.9974 \\  
0.50   & 0.65327  & 1.0701  & 0.9956  & 0.94077   & 1.1143  & 0.9954  & 1.16556  & 1.1441  & 0.9960 \\ 
0.60   & 0.70507  & 1.0814  & 0.9940  & 1.01844   & 1.1345  & 0.9936  & 1.26440  & 1.1713  & 0.9944 \\  
0.70   & 0.75056  & 1.0918  & 0.9922  & 1.08725   & 1.1539  & 0.9916  & 1.35252  & 1.1978  & 0.9925 \\  
0.80   & 0.79103  & 1.1016  & 0.9903  & 1.14897   & 1.1724  & 0.9893  & 1.43203  & 1.2236  & 0.9904 \\  
0.90   & 0.82740  & 1.1107  & 0.9882  & 1.20484   & 1.1902  & 0.9869  & 1.50442  & 1.2488  & 0.9880 \\[6pt]  
%
1.00   & 0.86033  & 1.1191  & 0.9861  & 1.25578   & 1.2071  & 0.9843  & 1.57080  & 1.2732  & 0.9855 \\* 
1.10   & 0.89035  & 1.1270  & 0.9839  & 1.30251   & 1.2232  & 0.9815  & 1.63199  & 1.2970  & 0.9828 \\  
1.20   & 0.91785  & 1.1344  & 0.9817  & 1.34558   & 1.2387  & 0.9787  & 1.68868  & 1.3201  & 0.9800 \\ 
1.30   & 0.94316  & 1.1412  & 0.9794  & 1.38543   & 1.2533  & 0.9757  & 1.74140  & 1.3424  & 0.9770 \\  
1.40   & 0.96655  & 1.1477  & 0.9771  & 1.42246   & 1.2673  & 0.9727  & 1.79058  & 1.3640  & 0.9739 \\  
1.50   & 0.98824  & 1.1537  & 0.9748  & 1.45695   & 1.2807  & 0.9696  & 1.83660  & 1.3850  & 0.9707 \\ 
1.60   & 1.00842  & 1.1593  & 0.9726  & 1.48917   & 1.2934  & 0.9665  & 1.87976  & 1.4052  & 0.9674 \\  
1.70   & 1.02725  & 1.1645  & 0.9703  & 1.51936   & 1.3055  & 0.9633  & 1.92035  & 1.4247  & 0.9640 \\* 
1.80   & 1.04486  & 1.1695  & 0.9680  & 1.54769   & 1.3170  & 0.9601  & 1.95857  & 1.4436  & 0.9605 \\*  
1.90   & 1.06136  & 1.1741  & 0.9658  & 1.57434   & 1.3279  & 0.9569  & 1.99465  & 1.4618  & 0.9570 \\[6pt]  
%
2.00   & 1.07687  & 1.1785  & 0.9635  & 1.59945   & 1.3384  & 0.9537  & 2.02876  & 1.4793  & 0.9534 \\*  
2.20   & 1.10524  & 1.1864  & 0.9592  & 1.64557   & 1.3578  & 0.9472  & 2.09166  & 1.5125  & 0.9462 \\  
2.40   & 1.13056  & 1.1934  & 0.9549  & 1.68691   & 1.3754  & 0.9408  & 2.14834  & 1.5433  & 0.9389 \\  
2.60   & 1.15330  & 1.1997  & 0.9509  & 1.72418   & 1.3914  & 0.9345  & 2.19967  & 1.5718  & 0.9316 \\  
2.80   & 1.17383  & 1.2052  & 0.9469  & 1.75794   & 1.4059  & 0.9284  & 2.24633  & 1.5982  & 0.9243 \\ 
3.00   & 1.19246  & 1.2102  & 0.9431  & 1.78866   & 1.4191  & 0.9224  & 2.28893  & 1.6227  & 0.9171 \\     
3.50   & 1.23227  & 1.2206  & 0.9343  & 1.85449   & 1.4473  & 0.9081  & 2.38064  & 1.6761  & 0.8995 \\     
4.00   & 1.26459  & 1.2287  & 0.9264  & 1.90808   & 1.4698  & 0.8950  & 2.45564  & 1.7202  & 0.8830 \\*   
4.50   & 1.29134  & 1.2351  & 0.9193  & 1.95248   & 1.4880  & 0.8830  & 2.51795  & 1.7567  & 0.8675 \\*    
5.00   & 1.31384  & 1.2402  & 0.9130  & 1.98981   & 1.5029  & 0.8721  & 2.57043  & 1.7870  & 0.8533 \\[6pt]    
%
6.00   & 1.34955  & 1.2479  & 0.9021  & 2.04901   & 1.5253  & 0.8532  & 2.65366  & 1.8338  & 0.8281 \\*     
7.00   & 1.37662  & 1.2532  & 0.8932  & 2.09373   & 1.5411  & 0.8375  & 2.71646  & 1.8673  & 0.8069 \\    
8.00   & 1.39782  & 1.2570  & 0.8858  & 2.12864   & 1.5526  & 0.8244  & 2.76536  & 1.8920  & 0.7889 \\     
9.00   & 1.41487  & 1.2598  & 0.8796  & 2.15661   & 1.5611  & 0.8133  & 2.80443  & 1.9106  & 0.7737 \\     
10.00  & 1.42887  & 1.2620  & 0.8743  & 2.17950   & 1.5677  & 0.8039  & 2.83630  & 1.9249  & 0.7607 \\     
12.00  & 1.45050  & 1.2650  & 0.8658  & 2.21468   & 1.5769  & 0.7887  & 2.88509  & 1.9450  & 0.7397 \\     
14.00  & 1.46643  & 1.2669  & 0.8592  & 2.24044   & 1.5828  & 0.7770  & 2.92060  & 1.9581  & 0.7236 \\     
16.00  & 1.47864  & 1.2683  & 0.8541  & 2.26008   & 1.5869  & 0.7678  & 2.94756  & 1.9670  & 0.7109 \\*     
18.00  & 1.48830  & 1.2692  & 0.8499  & 2.27556   & 1.5898  & 0.7603  & 2.96871  & 1.9734  & 0.7007 \\*     
20.00  & 1.49613  & 1.2699  & 0.8464  & 2.28805   & 1.5919  & 0.7542  & 2.98572  & 1.9781  & 0.6922 \\[6pt]     
%
25.00  & 1.51045  & 1.2710  & 0.8400  & 2.31080   & 1.5954  & 0.7427  & 3.01656  & 1.9856  & 0.6766 \\*     
30.00  & 1.52017  & 1.2717  & 0.8355  & 2.32614   & 1.5973  & 0.7348  & 3.03724  & 1.9898  & 0.6658 \\     
35.00  & 1.52719  & 1.2721  & 0.8322  & 2.33719   & 1.5985  & 0.7290  & 3.05207  & 1.9924  & 0.6579 \\     
40.00  & 1.53250  & 1.2723  & 0.8296  & 2.34552   & 1.5993  & 0.7246  & 3.06321  & 1.9942  & 0.6519 \\    
50.00  & 1.54001  & 1.2727  & 0.8260  & 2.35724   & 1.6002  & 0.7183  & 3.07884  & 1.9962  & 0.6434 \\     
60.00  & 1.54505  & 1.2728  & 0.8235  & 2.36510   & 1.6007  & 0.7140  & 3.08928  & 1.9974  & 0.6376 \\     
80.00  & 1.55141  & 1.2730  & 0.8204  & 2.37496   & 1.6013  & 0.7085  & 3.10234  & 1.9985  & 0.6303 \\    
100.00 & 1.55525  & 1.2731  & 0.8185  & 2.38090   & 1.6015  & 0.7052  & 3.11019  & 1.9990  & 0.6259 \\*    
200.00 & 1.56298  & 1.2732  & 0.8146  & 2.39283   & 1.6019  & 0.6985  & 3.12589  & 1.9998  & 0.6170 \\*    
\infty & 1.57080  & 1.2732  & 0.8106  & 2.40483   & 1.6020  & 0.6917  & 3.14159  & 2.0000  & 0.6079 \\[3pt] 
\end{longtable}
}


%%% Bibliography (biblatex)  %%%%%%%%%%%%%%%%%%%%%%%%%%%%%%%%%%%%%%%%%%%%%%%%%%%%%%%%%%%%%%%%%%%%%%

\defbibheading{bibintoc}{\chapter*{#1}\addcontentsline{toc}{backmatter}{\refname}} 
% this sets the title of contents name for bibliography to \refname (= References)
% change "backmatter" to "chapter" if you prefer a bold face entry in the table of contents

\printbibliography[title={\refname},heading=bibintoc]

% biblatex also supports chapter-by-chapter bibliography, https://tex.stackexchange.com/a/296502/119566
% see the biblatex manual, section 3.14.3


%%%% Option for natbib %%%%%%%%%%%%%

%%   use an appropriate style (.bst) and your own .bib file[s]

%\bibliographystyle{plainnat}
%\bibliography{mitthesis-sample.bib}

\end{document} 
 